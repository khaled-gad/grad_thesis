\documentclass[12pt]{article}

\usepackage{amsmath}
\usepackage{graphicx}
\usepackage{float}
\usepackage{listings}
\usepackage{xcolor}
\usepackage{enumitem}
\usepackage{booktabs}
\usepackage{hyperref}
\usepackage{amssymb}
\usepackage{tikz}
\usepackage{pgfplots}
\usepackage{booktabs}
\usepackage{xcolor}
\usepackage{url}

\usepackage{longtable}
\usepackage{booktabs}
\usepackage{multirow}

\setlength{\parindent}{0pt}      % no paragraph indent
\setlength{\parskip}{0.4em}      % vertical space between paragraphs

% Configure code listings
\lstset{
    language=C,
    basicstyle=\ttfamily\small,
    keywordstyle=\color{blue},
    commentstyle=\color{green!50!black},
    numbers=left,
    numberstyle=\tiny,
    frame=single,
    breaklines=true,
    captionpos=b
}
\usepackage[a4paper,margin=2.5cm]{geometry}
\geometry{margin=1in}

\begin{document}

%%%%%%%%%%%%%%%%%%%%%%%% Title Page %%%%%%%%%%%%%%%%%%%%%%%%%%%%%
\begin{titlepage}
    \centering
    \vspace{-2cm}
    \includegraphics[trim=4cm 8cm 4cm 12cm, clip, width=\textwidth]{figures/ejustlogo.png}
    \vspace{0.2cm}
    \textbf{\LARGE Optimizing Machine Learning Kernels Using RISC-V Vector Extension}\\[0.6cm]
    \LARGE CSE 500: Graduation Project 2 Thesis\\[1cm]

    \begin{center}
    \Large 
    \begin{tabular}{l l}
    \textbf{Name} & \textbf{University ID} \\
    Sally Reda Eldosouky Zeineldeen      & 120210008 \\
    Khaled Ahmed Anis Gad    & 120210024 \\
    Omar Tarek Ahmed Aly    & 120210099 \\
    Yousif Ibrahim Mohammed Masoud      & 120210281 \\
    Ebrahim Osama Shawky Tawfek    & 120210308 \\
    Ali Abdelkader Ali Elsawy   & 120210366 \\
    Mohammed Ashraf Abdelmonem Moawad   & 120210376 \\
    \end{tabular}
    \end{center}
    \centering
    \vspace{0.6cm}
    \textbf{\Large Mentored by}\\[0.4cm]
    \includegraphics[clip, width=0.4\textwidth]{figures/logo_si-vision_v3.png}
    
    \vspace{0.8cm}
    \Large Supervised by\\[0.6cm]
    \textbf{\Large Dr. Rami Zewail}\\[0.3cm]
    \textbf{\Large Prof. Mohammed S. Sayed}


\end{titlepage}
%%%%%%%%%%%%%%%%%%%%%%%%%%%%%%%%%%%%%%%%%%%%%%%%%%%%%%%%%%%%%%%%%

\tableofcontents
\newpage

\listoffigures
\listoftables
\newpage

% =============================================================================
% Abstract
% =============================================================================

\section*{Abstract}
\addcontentsline{toc}{section}{Abstract}

The proliferation of machine learning (ML) workloads across edge devices and embedded systems has intensified the demand for energy-efficient, high-performance computing architectures beyond conventional GPU-dominated paradigms. RISC-V, an open-source instruction set architecture, has emerged as a promising alternative, with its Vector Extension (RVV) offering scalable data-level parallelism suitable for computationally intensive ML operations. However, the literature reveals a significant gap in comprehensive, production-ready implementations of vectorized ML kernels optimized specifically for RVV 1.0, along with limited empirical benchmarking across diverse kernel categories on actual vector processor implementations. This study addresses these gaps by developing RVV64\_Library, a comprehensive collection of RISC-V Vector-accelerated kernels targeting deep learning and scientific computing workloads. The research objectives include implementing optimized vectorized versions of fundamental ML primitives---encompassing matrix multiplication, convolution, transposed convolution, dense (fully connected) layers, batch normalization, activation functions (ReLU, Leaky ReLU, Softmax), max pooling, bias addition, tensor arithmetic, and ONNX-style indexing operations (Gather, Scatter, Non-Maximum Suppression)---while systematically evaluating performance across different LMUL (Length Multiplier) configurations (M1, M2, M4, M8). The methodology employs C++ implementations utilizing RVV 1.0 intrinsics, with functional correctness validated against ONNX golden references using QEMU emulation, and performance benchmarked on the Ara vector co-processor, an open-source implementation of a scalable RISC-V vector unit. The library architecture emphasizes modularity through reusable low-level vector wrappers for loads, stores, reductions, multiply-accumulate, and mask operations, enabling clean and maintainable kernel implementations. Python bindings via shared libraries further enhance accessibility for rapid experimentation. Performance evaluation on the Ara co-processor demonstrates substantial speedups ranging from 4$\times$ to over 70$\times$ compared to scalar baseline implementations. Notably, matrix multiplication achieves up to 70.27$\times$ improvement using unrolled vectorization strategies, while activation and normalization kernels such as Leaky ReLU, batch normalization, and max pooling achieve speedups between 20$\times$ and 36$\times$. Compute-intensive linear operators and pointwise arithmetic kernels consistently demonstrate significant acceleration across all tested configurations. The library's practical applicability is validated through complete end-to-end inference implementations of LeNet-5 for digit classification and Tiny-YOLOv2 for object detection, demonstrating seamless integration of vectorized kernels into real neural network pipelines. This work establishes that the RISC-V Vector Extension, when properly optimized, provides a viable, high-performance, and energy-efficient alternative for accelerating ML inference on resource-constrained embedded platforms.

\vspace{1em}
\noindent\textbf{Keywords:} RISC-V Vector Extension (RVV), Machine Learning Kernel Acceleration, Vector Co-processor, Deep Learning Inference, High-Performance Embedded Computing

\newpage

% =============================================================================
% Acknowledgment
% =============================================================================
% !TEX root = ../Thesis.tex
\section*{Acknowledgment}
We would like to express our sincere gratitude to all those who contributed to the successful completion of this thesis. First and foremost, 
we extend our deepest appreciation to our academic supervisors, \textbf{Dr. Rami Zewail} and \textbf{Prof. Mohammed S. Sayed}, from the School of Electronics, 
Communications and Computer Engineering (ECCE) at Egypt-Japan University of Science and Technology (E-JUST), for their invaluable guidance, 
constructive feedback, and continuous support throughout this research. Their expertise and mentorship were instrumental in shaping the direction 
of this work. We are also profoundly grateful to \textbf{Si-Vision} company for providing the opportunity to conduct this project under their mentorship 
program, and we extend special thanks to \textbf{Eng. Youssef M. Fathy} for his dedicated supervision, technical insights, and unwavering encouragement. 
Finally, we acknowledge the Faculty of Engineering at E-JUST for providing the academic environment and resources that made this research possible.


% =============================================================================
% List of Abbreviations
% =============================================================================
\newpage
% !TEX root = ../Thesis.tex
\section*{List of Abbreviations}
\addcontentsline{toc}{section}{List of Abbreviations}

\small
\renewcommand{\arraystretch}{1.15}

\begin{longtable}{p{2.4cm} p{5.2cm} p{2.4cm} p{5.2cm}}
\toprule
\textbf{Abbreviation} & \textbf{Description} &
\textbf{Abbreviation} & \textbf{Description} \\
\midrule
\endfirsthead

\toprule
\textbf{Abbreviation} & \textbf{Description} &
\textbf{Abbreviation} & \textbf{Description} \\
\midrule
\endhead

ADAS & Advanced Driver Assistance Systems &
AI & Artificial Intelligence \\

ANN & Approximate Nearest Neighbor &
API & Application Programming Interface \\

ASIC & Application-Specific Integrated Circuit &
AVL & Application Vector Length \\

AVX & Advanced Vector Extensions &
AXI & Advanced eXtensible Interface \\

BERT & Bidirectional Encoder Representations from Transformers &
CHI & Coherent Hub Interface \\

CLINT & Core-Local Interrupt Controller &
CNN & Convolutional Neural Network \\

CPU & Central Processing Unit &
CSR & Control and Status Register \\

CUDA & Compute Unified Device Architecture &
CV32E40X & CORE-V 32-bit Embedded Processor 40X \\

CVA6 & CORE-V Application-class 64-bit Processor &
DMA & Direct Memory Access \\

DNN & Deep Neural Network &
DP & Double Precision \\

DPI & Direct Programming Interface &
DSP & Digital Signal Processing \\

ELF & Executable and Linkable Format &
ELEN & Element Length \\

FD-SOI & Fully Depleted Silicon On Insulator &
FFT & Fast Fourier Transform \\

FIR & Finite Impulse Response &
FLOPS & Floating-Point Operations Per Second \\

FMA & Fused Multiply-Add &
FO4 & Fan-Out of 4 \\

FP16 & 16-bit Floating Point &
FP32 & 32-bit Floating Point \\

FP64 & 64-bit Floating Point &
FPGA & Field-Programmable Gate Array \\

FPU & Floating-Point Unit &
GCC & GNU Compiler Collection \\

GDB & GNU Debugger &
GELU & Gaussian Error Linear Unit \\

GEMM & General Matrix Multiply &
GFLOPS & Giga Floating-Point Operations Per Second \\

GPT & Generative Pre-trained Transformer &
GPU & Graphics Processing Unit \\

HPC & High-Performance Computing &
HNSW & Hierarchical Navigable Small World \\

ID & Instruction Decode &
IEEE & Institute of Electrical and Electronics Engineers \\

IF & Instruction Fetch &
IIR & Infinite Impulse Response \\

INT8 & 8-bit Integer &
INT16 & 16-bit Integer \\

INT32 & 32-bit Integer &
INT64 & 64-bit Integer \\

IoU & Intersection over Union &
ISA & Instruction Set Architecture \\

ISS & Instruction Set Simulator &
IVFFLAT & Inverted File Flat \\

IVFPQ & Inverted File Product Quantization &
LLVM & Low Level Virtual Machine \\

LMUL & Length Multiplier &
LRU & Least Recently Used \\

LUT & Look-Up Table &
M1 & LMUL = 1 \\

M2 & LMUL = 2 &
M4 & LMUL = 4 \\

M8 & LMUL = 8 &
MAC & Multiply-Accumulate \\

MASKU & Mask Unit &
MCP & Model Context Protocol \\

MNIST & Modified National Institute of Standards and Technology &
NEON & Advanced SIMD Extension (ARM) \\

NMS & Non-Maximum Suppression &
OCR & Optical Character Recognition \\

ONNX & Open Neural Network Exchange &
OpenCL & Open Computing Language \\

OVI & Open Virtual Interface &
PCIe & Peripheral Component Interconnect Express \\

PLIC & Platform-Level Interrupt Controller &
PPA & Power, Performance, and Area \\

QEMU & Quick Emulator &
ReLU & Rectified Linear Unit \\

RISC-V & Reduced Instruction Set Computer – Five &
RTL & Register Transfer Level \\

RVV & RISC-V Vector Extension &
RVV64 & RISC-V Vector Extension 64-bit \\

SEW & Selected Element Width &
SIMD & Single Instruction, Multiple Data \\

SLDU & Slide Unit &
SNR & Signal-to-Noise Ratio \\

SRAM & Static Random-Access Memory &
SVG & Scalable Vector Graphics \\

TLM & Transaction-Level Modeling &
TSVC & Test Suite for Vectorizing Compilers \\

UVM & Universal Verification Methodology &
VALU & Vector Arithmetic Logic Unit \\

VELEM & Vector Element Unit &
VEU & Vector Execution Unit \\

VFPU & Vector Floating-Point Unit &
VLA & Vector-Length Agnostic \\

VLEN & Vector Length &
VLS & Vector Length Specific \\

VLSU & Vector Load/Store Unit &
VMUL & Vector Multiplier \\

VOC & Visual Object Classes &
VP & Virtual Prototype \\

VPIPE\_W & Vector Pipeline Width &
VQ & Vector Queue \\

VRF & Vector Register File &
VSLD & Vector Slide Unit \\

WCET & Worst-Case Execution Time &
XIF & eXtension Interface \\

YOLO & You Only Look Once & & \\

\bottomrule
\end{longtable}


% =============================================================================
% Introduction
% =============================================================================
\newpage
% !TEX root = ../Thesis.tex
\section{Introduction}

\subsection{Motivation}

The rapid evolution of artificial intelligence (AI) and digital signal processing (DSP) applications has fundamentally transformed the computational requirements of modern computing systems. AI workloads, particularly deep learning models, demand massive parallel computation for operations such as matrix multiplication, convolution, and tensor operations. Similarly, DSP applications require intensive mathematical operations including filtering, Fourier transforms, and correlation analysis. These computational patterns share a common characteristic: they involve highly parallel, data-intensive operations that can benefit significantly from vectorized execution.

% FLOPs growth figure here
\begin{figure}[H]
    \centering
    \includegraphics[width=0.85\textwidth]{figures/Flops.png}
    \caption{Exponential growth in AI model computational requirements over time, showing the dramatic increase in FLOPs required for training state-of-the-art models.}
    \label{fig:ai-growth}
\end{figure}

Traditional scalar processors, designed primarily for sequential instruction execution, face significant challenges when processing these data-parallel workloads. The von Neumann architecture, with its single instruction stream operating on individual data elements, creates a fundamental bottleneck for AI and DSP applications. For example, a typical convolution operation in a convolutional neural network (CNN) involves millions of multiply-accumulate operations that could theoretically be executed in parallel, but scalar processors must execute them sequentially, leading to substantial performance degradation.

The performance gap becomes even more pronounced when considering the memory bandwidth requirements of AI and DSP applications. These workloads often involve large datasets that exceed the capacity of processor caches, leading to frequent memory accesses. The arithmetic intensity---the ratio of computation to memory access---of many AI and DSP kernels is relatively low, meaning that processors spend significant time waiting for data rather than performing useful computation. This phenomenon, commonly referred to as the ``memory wall,'' represents a fundamental challenge for data-intensive computing.

\subsection{Limitations of Current Solutions}

Current solutions to these challenges have primarily relied on specialized hardware architectures and proprietary vector processing extensions. Graphics Processing Units (GPUs) have become the de facto standard for AI acceleration due to their thousands of parallel cores designed for data-parallel computation. However, GPUs present several limitations for AI and DSP applications:

\begin{itemize}
    \item \textbf{Power consumption}: GPUs consume significant power, making them unsuitable for edge computing and mobile applications where energy efficiency is paramount.
    \item \textbf{Programming complexity}: The GPU programming model, while powerful, requires specialized knowledge (CUDA, OpenCL) and often results in complex code that is difficult to optimize and maintain.
    \item \textbf{Integration challenges}: GPUs are discrete components requiring separate memory spaces and PCIe communication, introducing latency and bandwidth constraints for certain workloads.
\end{itemize}

Proprietary vector processing solutions, such as Intel's Advanced Vector Extensions (AVX) and ARM's NEON, provide another approach to accelerating data-parallel workloads. These extensions add vector processing capabilities to traditional CPU architectures, allowing multiple data elements to be processed with a single instruction (SIMD---Single Instruction, Multiple Data). However, these solutions have significant drawbacks that limit their effectiveness and adoption:

\begin{enumerate}
    \item \textbf{Vendor lock-in}: Proprietary vector extensions create situations where software optimized for one vendor's vector instructions cannot efficiently run on competitors' hardware, fragmenting the software ecosystem.
    
    \item \textbf{Fixed vector widths}: These extensions typically use fixed vector widths (e.g., 128-bit for NEON, 256-bit or 512-bit for AVX), meaning that software must be written for specific vector lengths and may not efficiently utilize processors with different vector capabilities.
    
    \item \textbf{Licensing costs}: Licensing costs and restrictions associated with proprietary architectures can be prohibitive, particularly for smaller companies and research institutions developing specialized AI and DSP applications.
    
    \item \textbf{Limited extensibility}: The closed nature of proprietary ISAs makes it difficult for researchers and developers to experiment with custom instructions or architectural modifications.
\end{enumerate}

\subsection{The RISC-V Vector Extension as a Solution}

RISC-V Vector Extensions (RVV) emerged as a promising solution to address these challenges by providing an open-source, royalty-free vector processing architecture specifically designed for data-parallel computation. Unlike proprietary alternatives, RVV is developed through an open, collaborative process that ensures the architecture meets the diverse needs of the computing community.

The most distinctive feature of RVV is its \textbf{vector-length agnostic (VLA)} programming model, which represents a fundamental departure from traditional fixed-width SIMD approaches. In conventional vector processing, software must be written for specific vector widths, and different code paths are often required to support processors with different vector capabilities. RVV's vector-length agnostic model allows the same code to run efficiently across processors with different vector lengths, from embedded systems with short vectors to high-performance computing systems with very long vectors.

This flexibility is particularly valuable for AI and DSP applications, which span a wide range of computing environments with different performance and power requirements. An AI inference algorithm written using RVV can run efficiently on both a power-constrained edge device with 128-bit vectors and a high-performance server processor with 2048-bit vectors, without requiring code modifications or recompilation. The ratification of RVV Version 1.0 in late 2021 provided a crucial guarantee of stability, signaling to the industry that the architecture was mature and ready for widespread adoption.

\subsection{Problem Statement}

Despite the architectural advantages of the RISC-V Vector Extension, developing optimized kernels for RVV remains a challenging task. New developers often face two major obstacles:

\begin{enumerate}
    \item \textbf{Lack of standardized workflows}: There is currently no unified methodology for vector kernel development that encompasses design, verification, and performance evaluation in a cohesive framework.
    
    \item \textbf{Limited guidance on verification and evaluation}: While performance measurement is essential to demonstrate the benefits of vectorization, ensuring functional correctness is equally critical, especially when kernels are applied in sensitive domains such as artificial intelligence or embedded systems.
\end{enumerate}

Furthermore, the absence of widely available RVV-capable silicon necessitates reliance on emulation and RTL simulation environments for development and validation. This creates additional complexity in establishing reproducible benchmarking methodologies that can provide meaningful performance insights.

\subsection{Research Objectives}

This thesis investigates the role and potential impact of RISC-V Vector Extensions in accelerating AI and DSP applications. The primary objectives of this research are:

\begin{enumerate}
    \item \textbf{Develop a systematic framework} for the design, verification, and performance evaluation of RISC-V vector kernels that integrates open-source tools into a reproducible workflow.
    
    \item \textbf{Implement optimized RVV kernels} for fundamental machine learning and DSP operations, including:
    \begin{itemize}
        \item Matrix operations: multiplication, transposition, addition
        \item Activation functions: ReLU, Softmax
        \item Convolutional layers
        \item Training kernels: linear regression gradient descent
    \end{itemize}
    
    \item \textbf{Establish functional verification methodologies} using ONNX (Open Neural Network Exchange) as a golden reference framework, ensuring correctness through quantitative metrics such as Signal-to-Noise Ratio (SNR) and Maximum Absolute Error.
    
    \item \textbf{Conduct cycle-accurate performance evaluation} using RTL simulation with the Ara and Vicuna vector coprocessors, quantifying the speedup achieved through vectorization over scalar implementations.
    
    \item \textbf{Analyze the trade-offs} between different architectural approaches (high-performance vs. embedded) and provide insights into optimal kernel design strategies for various deployment scenarios.
\end{enumerate}

\subsection{Research Contributions}

This thesis makes the following contributions to the field of RISC-V vector processing for machine learning and DSP applications:

\begin{enumerate}
    \item \textbf{A reproducible three-step framework} integrating kernel design using RVV C-intrinsics, functional verification against ONNX golden references, and cycle-accurate performance analysis using RTL simulation with Verilator. This framework provides a systematic methodology that can be adopted by other researchers and developers.
    
    \item \textbf{A library of optimized RVV kernels} (RVV64\_Library) implementing fundamental operations for neural network inference and signal processing. The library demonstrates efficient use of RVV features including strip-mining loops, vector length agnostic programming, LMUL configuration, and masked operations.
    
    \item \textbf{Comprehensive performance characterization} of vectorized kernels on two distinct RTL platforms:
    \begin{itemize}
        \item \textbf{Ara}: A high-performance 64-bit vector coprocessor targeting application-class workloads with configurable lane counts (2--16 lanes) and full floating-point support.
        \item \textbf{Vicuna}: A timing-predictable 32-bit vector coprocessor targeting embedded real-time systems with deterministic execution guarantees.
    \end{itemize}
    
    \item \textbf{Quantitative analysis} demonstrating significant speedups achieved through vectorization, with experimental results showing up to 3.6$\\times$ improvement for matrix multiplication and 2.6$\\times$ for ReLU activation compared to scalar implementations.
    
    \item \textbf{A containerized development environment} (Docker-based) ensuring reproducibility across different development systems and simplifying the onboarding process for future researchers working with RISC-V vector extensions.
\end{enumerate}

\subsection{Thesis Organization}

The remainder of this thesis is organized as follows to guide the reader from foundational concepts to our specific contributions:

\textbf{Chapter 2: Background and Related Work} establishes the necessary theoretical foundation, providing a detailed overview of the RISC-V ISA, its relevant extensions, and the architectural principles of the Vector (RVV) extension. This chapter reviews the vector-length agnostic programming model, control and status registers, and the rich instruction set that enables efficient data-parallel processing. Additionally, it examines foundational academic research on RVV for machine learning acceleration, including work on specialized processors and prior vectorization efforts.

\textbf{Chapter 3: Development Tools} outlines the practical tools and workflows established for implementation and validation. This includes the RISC-V GNU toolchain configuration, QEMU emulator setup for functional testing, Ara RTL simulation environment, Docker containerization for reproducible development, and ONNX framework integration for functional verification.

\textbf{Chapter 4: Methodology} presents the systematic three-step framework that forms the core of this research: (1) kernel design using RVV C-intrinsics with vector-length agnostic programming, (2) functional verification against ONNX golden references using quantitative metrics (SNR and MaxAbs), and (3) cycle-accurate performance evaluation using RTL simulation with Verilator. This chapter details the kernel selection criteria, vectorization design approach, verification workflow, and performance measurement techniques.

\textbf{Chapter 5: Library} introduces the RVV64\_Library, our collection of optimized RISC-V vector kernels for machine learning and DSP applications. This chapter systematically presents each implemented kernel---including matrix operations (multiplication, transposition), activation functions (ReLU, Softmax), convolutional layers, and training kernels (linear regression)---with detailed explanations of the naive sequential approach, proposed RVV-based solution, and performance optimization strategies.

\textbf{Chapter 6: Hardware (RTL Cores)} details the two RTL platforms used for cycle-accurate performance evaluation: Ara and Vicuna. This chapter explains the role of RTL simulation in architectural research, describes the microarchitectural details of each coprocessor (Ara for high-performance throughput-oriented workloads and Vicuna for timing-predictable embedded systems), and justifies their selection as representative platforms spanning the full spectrum of RISC-V vector implementations.

\textbf{Chapter 7: Results and Discussion} presents both theoretical analysis and empirical evaluation of the implemented kernels. This chapter provides complexity analysis, theoretical speedup calculations, and experimental results from RTL simulation on both Ara and Vicuna platforms. Quantitative performance improvements are analyzed, including achieved speedups, functional unit utilization, and the impact of different LMUL configurations. The discussion interprets these results in the context of ML/DSP application deployment scenarios.

\textbf{Chapter 8: Conclusion} summarizes the key findings of this research, reflecting on the demonstrated effectiveness of the RISC-V Vector Extension for accelerating data-parallel workloads. This chapter synthesizes the contributions made through our systematic framework, optimized kernel library, and comprehensive performance characterization across diverse hardware platforms.

\textbf{Chapter 9: Future Work} outlines promising directions for extending this research, including algorithmic expansion to additional ML operators (pooling, normalization, attention mechanisms), hardware-level validation on FPGA and ASIC platforms, integration with established ML frameworks (TensorFlow Lite, ONNX Runtime), and exploration of multi-core vector configurations for improved scalability.

Supporting materials are provided in the final sections, including acknowledgments of collaborators and advisors, a comprehensive reference list, and complete source code listings for all implemented kernels in the appendix.


% =============================================================================
% Background & Related Work
% =============================================================================
\newpage
\section{Background \& Related Work}

\subsection{RISC-V Architecture Overview}

RISC-V (pronounced ``risk-five'') is an open-source instruction set architecture (ISA) that has revolutionized processor design by providing a free, extensible alternative to proprietary architectures. Developed at the University of California, Berkeley, beginning in 2010, RISC-V was created to address fundamental limitations in the processor industry, particularly the dominance of proprietary ISAs that created barriers to innovation and increased costs for processor development.

The development of RISC-V was motivated by several critical issues in the computing industry that had become increasingly problematic for AI and DSP applications. Traditional proprietary ISAs, such as x86 and ARM, require expensive licensing agreements that can be prohibitive for companies developing specialized processors for AI and DSP workloads. These licensing costs are particularly burdensome for startups and research institutions that want to experiment with novel architectural approaches.

RISC-V addresses these challenges through several fundamental design principles that make it particularly well-suited for AI and DSP applications:

\textbf{Open Source Philosophy:} RISC-V specifications are freely available under Creative Commons licenses, and anyone can implement, modify, or extend RISC-V processors without paying royalties or obtaining permission. This openness eliminates one of the major barriers to innovation in processor design and enables a diverse ecosystem of implementations tailored for specific applications.

\textbf{Modular Architecture:} RISC-V follows a modular design philosophy where a minimal base integer instruction set is supplemented by optional standard extensions. This modularity is particularly valuable for AI and DSP processors, which can include only the extensions needed for their specific applications, reducing implementation complexity and cost.


\textbf{Scalability Across Application Domains:} RISC-V supports multiple data widths (32-bit, 64-bit, and 128-bit) and can scale from microcontrollers to high-performance processors. This scalability is crucial for AI and DSP applications, which span a wide range of computing environments from embedded edge devices to high-performance computing clusters.

\begin{figure}[H]
    \centering
    \includegraphics[width=0.8\textwidth]{figures/growth.png}
    \caption{RISC-V ecosystem growth. Source: RISC-V International}
    \label{ RISC-V ecosystem growth. Source: RISC-V International}
\end{figure}    


\subsection{RISC-V Extensions for Machine Learning}

The extensible nature of RISC-V is fundamental to its success in AI and DSP applications, allowing specialized functionality to be added to the base instruction set through a well-defined extension mechanism. This extensibility enables processors to be tailored for specific application domains while maintaining compatibility with the broader RISC-V ecosystem.

\subsubsection{Standard Extensions}

\textbf{M Extension (Integer Multiplication and Division):} The M extension adds integer multiplication, division, and remainder operations that are fundamental for many AI and DSP algorithms. In AI applications, integer multiplication is crucial for quantized neural networks that use integer arithmetic instead of floating-point operations to reduce power consumption and increase performance.

\textbf{F Extension (Single-Precision Floating-Point):} The F extension provides IEEE 754 single-precision (32-bit) floating-point arithmetic, which is the most commonly used precision for AI training and many DSP applications. The extension includes fused multiply-add (FMA) instructions that are particularly important for AI and DSP workloads, as convolution operations in neural networks consist primarily of multiply-accumulate patterns that can be efficiently implemented using FMA instructions.

\textbf{D Extension (Double-Precision Floating-Point):} The D extension adds IEEE 754 double-precision (64-bit) floating-point arithmetic, which is important for AI training applications that require higher numerical precision and certain DSP applications that need extended dynamic range.

\textbf{V Extension (Vector Operations):} The V extension is the most significant addition to RISC-V for AI and DSP applications, providing comprehensive support for data-parallel vector operations. This extension represents a fundamental advancement in vector processing architecture and is the primary focus of this work.

\subsection{The RISC-V Vector Extension}

The RISC-V Vector (RVV) Extension stands out as one of the most consequential developments for modern computing workloads. Unlike traditional Single Instruction, Multiple Data (SIMD) architectures that operate on fixed-size registers, RVV was designed with a philosophy of flexibility, scalability, and efficiency achieved through novel architectural concepts.

\subsubsection{Architectural Principles}

\textbf{Vector Registers and Configuration:} The V extension introduces 32 vector registers (\texttt{v0}--\texttt{v31}). The core architectural parameter is VLEN (Vector Length), which specifies the length of these registers in bits. VLEN is an implementation-defined choice, not fixed by the specification, and can range from small values (e.g., 128 bits) for embedded systems to very large values (e.g., 4096 bits or more) for supercomputers. Another key parameter is ELEN (Element Length), which is the maximum size of a single data element that can be processed.

\textbf{Vector Control and Status Registers (CSRs):} The power and flexibility of the V extension are managed through key Control and Status Registers:

\begin{itemize}
    \item \texttt{vtype}: Configures the vector unit for subsequent operations by setting the selected element width (\texttt{vsew}), vector length multiplier (\texttt{vlmul}) for register grouping, and behavior controls for tail and masked-out elements (\texttt{vta}/\texttt{vma}).
    \item \texttt{vl}: Set by the programmer to specify the number of elements to process in upcoming vector instructions, ranging from 0 to a hardware-dependent maximum.
    \item \texttt{vlenb}: A read-only register that reports the hardware's vector register length (VLEN) in bytes.
\end{itemize}

\textbf{Vector-Length Agnostic (VLA) Execution:} The combination of the \texttt{vsetvli} instruction and the \texttt{vl} register enables RVV's most powerful feature: Vector-Length Agnosticism. Unlike fixed-length SIMD (e.g., Intel's AVX or ARM's NEON), where code is written for a specific vector width, VLA code is portable across any hardware implementation, regardless of its VLEN.

The typical execution flow follows a ``strip-mining'' pattern:
\begin{enumerate}
    \item A programmer has a large array of N elements to process
    \item The code enters a loop and calls \texttt{vsetvli}, passing the remaining number of elements
    \item The hardware automatically sets \texttt{vl} to the minimum of the requested number and the maximum it can physically handle (VMAX), configuring \texttt{vtype} appropriately
    \item Subsequent vector instructions operate on \texttt{vl} elements
    \item The loop continues, processing chunks of data until all N elements are complete
\end{enumerate}

This approach means a single compiled binary can run with optimal efficiency on both a low-power microcontroller with VLEN=128 and a high-performance compute node with VLEN=4096, without requiring recompilation or code modification.

\textbf{Rich and Orthogonal Instruction Set:} The V extension provides a comprehensive set of instructions orthogonal to data types, allowing the same opcodes to work on integers and floats of different widths as configured by \texttt{vtype}. Key instruction categories include:

\begin{itemize}
    \item \textbf{Vector Arithmetic:} Integer, fixed-point, and floating-point operations
    \item \textbf{Vector Memory Access:} Unit-stride (contiguous), strided (every Nth element), and indexed scatter/gather operations
    \item \textbf{Vector Permutation:} Instructions for shuffling data within and between vector registers
    \item \textbf{Masking and Predication:} Most vector instructions can be masked, performing operations only on elements where a corresponding bit in mask register \texttt{v0} is set
    \item \textbf{Reduction Operations:} Built-in support for combining all vector elements into a scalar result (sum, min, max, logical reductions)
\end{itemize}

\subsubsection{Programming with RISC-V Vector Intrinsics}

While assembly language provides direct control over vector instructions, RISC-V vector intrinsics offer a more maintainable and portable approach to vectorized programming. Intrinsics are C/C++ functions that map directly to vector instructions, providing the performance benefits of assembly with the readability and toolchain integration of high-level languages.

\textbf{Advantages of Intrinsics:}
\begin{itemize}
    \item \textbf{Compiler Integration:} Intrinsics work seamlessly with standard C/C++ compilers, enabling better optimization, register allocation, and instruction scheduling
    \item \textbf{Type Safety:} Unlike inline assembly, intrinsics are type-checked by the compiler, catching errors at compile time
    \item \textbf{Portability:} Code using intrinsics can be more easily ported across different RISC-V implementations
    \item \textbf{Maintainability:} Intrinsic-based code is more readable and easier to debug than raw assembly
\end{itemize}

\textbf{Common Intrinsic Patterns:}

\textit{Setting Vector Length:}
\begin{lstlisting}[language=C++]
size_t vl = __riscv_vsetvl_e32m1(n);
\end{lstlisting}

\textit{Vector Load/Store:}
\begin{lstlisting}[language=C++]
vfloat32m1_t v = __riscv_vle32_v_f32m1(ptr, vl);
__riscv_vse32_v_f32m1(ptr, v, vl);
\end{lstlisting}

\textit{Arithmetic Operations:}
\begin{lstlisting}[language=C++]
v_result = __riscv_vfadd_vv_f32m1(v1, v2, vl);
v_result = __riscv_vfmul_vf_f32m1(v, scalar, vl);
\end{lstlisting}

\textit{Fused Multiply-Accumulate:}
\begin{lstlisting}[language=C++]
v_acc = __riscv_vfmacc_vv_f32m1(v_acc, v1, v2, vl);
\end{lstlisting}

\textit{Reduction Operations:}
\begin{lstlisting}[language=C++]
vfloat32m1_t v_sum = __riscv_vfredsum_vs_f32m1_f32m1(v, v_zero, vl);
float sum = __riscv_vfmv_f_s_f32m1_f32(v_sum);
\end{lstlisting}

\subsection{Ara RISC-V Vector Coprocessor}

Ara is a scalable vector coprocessor developed at ETH Zurich that works alongside the CVA6 (formerly Ariane) scalar core to accelerate RISC-V vector operations. It supports 2--16 parallel lanes and implements the RV64GCV instruction set. The processor uses a lane-based design where each lane handles 64-bit wide operations. Vector length (VLEN) can be configured from 128 to 1024 bits depending on application needs. It achieves up to 97\% FPU utilization when running a $256 \times 256$ double-precision matrix multiplication on sixteen lanes.
\begin{figure}[H]
    \centering
    \includegraphics[width=0.8\textwidth]{figures/ara.jpg}
    \caption{Top-level architecture diagram of the Ara RISC-V Vector Coprocessor}
    \label{fig:ara_architecture}
\end{figure}
\subsubsection{Architecture Components}

\textbf{Lane Design:} Each lane operates as an independent vector processing unit with its own ALU and FPU. The lanes process different parts of vector data in parallel, with each lane handling vector elements based on its position. For example, in a 4-lane configuration, lane 0 processes elements 0, 4, 8, 12, while lane 1 handles elements 1, 5, 9, 13.

\textbf{Sequencer:} The sequencer acts as the control center for instruction dispatch and execution tracking, maintaining a global view of all vector instructions across lanes. It can track up to 8 parallel vector instructions and ensures correct execution order while interfacing with the CVA6 scalar core to coordinate vector and scalar instruction execution.

\textbf{Vector Register File (VRF):} The VRF features eight 64-bit wide banks per lane, providing 512 bits of total bandwidth per lane. Banks are accessed in parallel to supply operands to multiple units simultaneously. When multiple functional units need to access the same bank, conflicts are resolved dynamically with a weighted round-robin arbiter. The initial bank of each vector register is shifted in a ``barber's pole'' fashion to avoid banking conflicts when functional units fetch the first element of different vector registers.

\textbf{Slide Unit (SLDU):} The SLDU handles vector element rearrangement operations essential for many algorithms, performing vector slides, element insertion and extraction, and vector shuffles. It must access all VRF banks simultaneously for some operations.

\textbf{Vector Load/Store Unit (VLSU):} The VLSU is responsible for all vector memory operations and includes sophisticated address generation capabilities. It handles multiple outstanding memory requests and includes logic for address calculation for different stride patterns, memory request coalescing for unit-stride operations, and memory ordering constraint handling.

\textbf{Queue Management:} The multi-banked organization of the VRF can lead to banking conflicts when several functional units try to access operands in the same bank. Each lane has a set of operand queues between the VRF and the functional units to absorb such banking conflicts.

\subsubsection{Ara as a Performance Reference}

Ara serves as the hardware reference platform for this project, providing cycle-accurate performance measurements of our vectorized kernels. By compiling and running our implementations on Ara's synthesizable RTL model, we can obtain realistic performance metrics including:

\begin{itemize}
    \item \textbf{Cycle Counts:} Precise measurement of execution cycles for both vectorized and scalar implementations
    \item \textbf{Speedup Analysis:} Quantitative comparison showing the performance benefits of vectorization
    \item \textbf{Hardware Utilization:} Insights into how effectively our code uses the vector processing resources
    \item \textbf{Energy Efficiency:} Understanding of the power-performance trade-offs in our implementations
\end{itemize}

\subsection{Foundational Research on RVV for Machine Learning}

Several academic research projects have validated the theoretical benefits of RVV and explored microarchitectural techniques to further accelerate specific workloads, particularly Deep Neural Networks (DNNs) and machine learning algorithms.

\subsubsection{SPEED: Scalable Multi-Precision DNN Processor}

Wang et al.\ addressed the gap between general-purpose vector processors and specialized DNN inference demands in their work on SPEED. They identified that standard RVV processors struggle with modern DNNs due to limited support for low-precision data types (e.g., 4-bit), constrained computational throughput for massive MAC operations, and inefficient dataflows that underutilize hardware.

SPEED integrates a highly parameterized Systolic Array Unit (SAU) within each vector lane, acting as a dedicated matrix multiplication engine working in concert with the standard RVV ALU. The architecture is enhanced with custom instructions (VSACFG, VSALD, VSAM) to manage the SAU and a specialized dataflow strategy to maximize data reuse and computational efficiency. When synthesized for 28nm technology and compared to the Ara processor, SPEED demonstrates significant improvement in area efficiency (2.04x for 16-bit and 1.63x for 8-bit operations).

\subsubsection{RVV Efficiency for ANN Algorithms}

Rumyantsev et al.\ presented a practical and theoretical analysis of applying RVV to accelerate Approximate Nearest Neighbor (ANN) search algorithms. Their work aimed to quantify performance gains from using RVV to optimize common ANN libraries like Faiss, Annoy, and NMSLIB, where distance computation is the primary bottleneck.

They implemented and optimized key algorithms (IVFFlat, IVFPQ, HNSW) using RVV intrinsics and benchmarked their performance against scalar code on a Lichee Pi 4A board. Experimental results showed that RVV-optimized code achieved speedups of up to 2.58x over scalar versions. The paper also presents a theoretical model of a parameterized vector unit used to determine optimal hardware configuration (register length, number of functional units) for this class of algorithms, providing both real-world performance data and theoretical hardware design insights.
\begin{figure}[H]
    \centering
    \includegraphics[width=0.8\textwidth]{figures/performance.png}
    \caption{Performance acceleration from RVV optimization for various ANN algorithms on the Epsilon and GloVe datasets}
    \label{fig:Performance acceleration from RVV optimization for various ANN algorithms on the Epsilon and
GloVe datasets}
\end{figure}

% =============================================================================
% Methodology: Architecture & Implementations
% =============================================================================
\newpage
\section{Methodology: Architecture \& Implementations}

This chapter describes the architectural and software methodology used to build the RaiVeX Library. It first motivates and details the selection of machine learning kernels, then presents the development toolchain and execution environment used for RVV-based implementation and verification. The chapter then explains the design of the vectorized kernels themselves, followed by the functional verification strategy and kernel-level correctness results, before concluding with model-level validation using complete LeNet-5 and Tiny-YOLOv2 inference pipelines.

% Deep Learning Kernel Selection and Justification (from methodology.tex)
% !TEX root = ../Thesis.tex
\subsection{Deep Learning Kernel Selection and Justification}

The development of the RVV64\_Library begins with careful selection of computational kernels that represent fundamental building blocks for machine learning inference and digital signal processing applications. The selection process follows a systematic prioritization strategy based on several criteria.

\subsubsection{Selection Criteria}

Kernels are selected based on the following criteria:

\begin{enumerate}
    \item \textbf{Computational significance}: Operations that constitute performance bottlenecks in target applications. Profiling studies of neural network inference workloads indicate that matrix multiplication and convolution operations account for 80-95\% of total execution time, making them priority targets for optimization.
    
    \item \textbf{Data parallelism potential}: Operations exhibiting high degrees of data-level parallelism that can be efficiently mapped to RVV's vector execution model. Element-wise operations and reduction operations naturally fit this category.
    
    \item \textbf{Memory access patterns}: Operations with predictable memory access patterns that can leverage RVV's unit-stride, strided, and indexed memory operations without excessive overhead.
    
    \item \textbf{Composability}: Fundamental operations that can be composed to build more complex computational graphs. For example, matrix multiplication and activation functions are basic building blocks for fully-connected neural network layers.
    
    \item \textbf{Algorithmic complexity}: A mix of simple and complex kernels to demonstrate RVV's versatility. Simple element-wise operations like ReLU provide straightforward vectorization examples, while operations like Softmax requiring normalization demonstrate handling of complex multi-phase algorithms.
\end{enumerate}

\subsubsection{Prioritized Kernel Categories}

Based on these criteria, kernels are organized into three priority tiers tailored to AI and computer vision workloads, targeting models such as LeNet-5 and Tiny-YOLOv2.

\textbf{Tier 1 (Core Linear Algebra for DNNs)}: Matrix multiplication, matrix addition, matrix transposition, and dot product. These operations form the backbone of convolutional and fully connected layers, dominating the computation in deep neural networks used for image classification and object detection.

\textbf{Tier 2 (Activation, Normalization, and Basic CV)}: ReLU, Sigmoid, Tanh, and Softmax activation functions, along with simple elementwise normalization and scaling. These operations are lighter than matrix kernels but are essential for expressing nonlinearities in CNNs like LeNet-5 and for stabilizing training and inference in models such as Tiny-YOLOv2.

\textbf{Tier 3 (Specialized Vision Operations)}: Convolution and pooling operations, as well as domain-specific kernels commonly used in computer vision pipelines (e.g., feature extraction, downsampling, and simple pre/post-processing). These operations showcase RVV's ability to handle complex multi-dimensional data access patterns typical of CNN-based AI workloads.


% Development Toolchain (from tools.tex)
% !TEX root = ../Thesis.tex
\subsection{Development Toolchain}

\subsubsection{RISC-V GNU Toolchain}

The RISC-V GNU Toolchain provides the foundational software infrastructure required to develop, compile, link, and debug programs targeting the RISC-V instruction set architecture. In this project, the toolchain serves as the primary means of compiling and validating RISC-V Vector Extension (RVV)–based kernel implementations prior to functional verification and RTL-level evaluation.

The toolchain was selected due to its mature support for the RISC-V ecosystem, active community maintenance, and integration of RVV 1.0 intrinsics within the GNU Compiler Collection (GCC). This enables the development of vectorized kernels using high-level C/C++ code while maintaining explicit control over generated vector instructions.

\paragraph{Toolchain Components}

\textbf{Compiler (\texttt{riscv64-unknown-linux-gnu-g++}):}  
The GCC compiler translates C/C++ source code into RISC-V assembly and object code. In this work, vectorization is primarily achieved through explicit RVV intrinsics rather than relying on automatic vectorization. This approach ensures deterministic mapping between source-level kernel implementations and the generated vector instructions, which is essential for functional verification and architectural analysis.

The following architecture-specific compilation flags are used throughout the project:
\begin{itemize}
    \item \texttt{-march=rv64gcv}: Enables the RV64GCV ISA, including the RISC-V Vector Extension
    \item \texttt{-mabi=lp64d}: Specifies the 64-bit ABI with double-precision floating-point support
    \item \texttt{-O2}, \texttt{-O3}: Enable standard compiler optimizations without altering the explicit vectorization strategy
\end{itemize}

\textbf{Assembler (\texttt{riscv64-unknown-linux-gnu-as}):}  
Converts RISC-V assembly code into object files, supporting both scalar and RVV instruction encodings. It is used indirectly through the compiler toolchain and for inspecting compiler-generated assembly during kernel analysis.

\textbf{Linker (\texttt{riscv64-unknown-linux-gnu-ld}):}  
Links object files and resolves external symbols to produce final executable binaries suitable for execution under emulation or simulation.

\textbf{Debugger (\texttt{riscv64-unknown-linux-gnu-gdb}):}  
Provides source-level debugging capabilities for RISC-V binaries executed under QEMU, including inspection of scalar and vector registers. This functionality is essential for validating intermediate kernel states during development.

\textbf{Binary Utilities:}  
Tools such as \texttt{objdump}, \texttt{readelf}, and \texttt{nm} are used to inspect binaries, verify RVV instruction generation, and analyze symbol information.

\textbf{Runtime Libraries:}  
The GNU C Library (\texttt{glibc}) is used for Linux-based execution under QEMU, providing standard runtime support required for user-mode and system-mode emulation. Bare-metal runtime support is not a primary focus of this work.

Collectively, the RISC-V GNU Toolchain constitutes the software development layer of the project, enabling portable kernel implementation and serving as the entry point for subsequent functional and RTL-level verification.

% ------------------------------------------------------------

\subsubsection{QEMU Emulator}

QEMU (Quick Emulator) is employed as the primary functional execution environment for RISC-V binaries in this project. QEMU provides architecturally correct emulation of the RISC-V ISA, including the RISC-V Vector Extension, and is used exclusively for functional correctness validation and software-level testing. Performance results reported in this thesis are derived from RTL simulation rather than QEMU execution.

QEMU was selected over alternative simulators due to its system-level completeness, integration with standard debugging tools, and practical support for executing complex software stacks.

\paragraph{Rationale for Using QEMU}

While Spike is the official RISC-V ISA simulator, QEMU offers several advantages aligned with the goals of this project. QEMU supports full Linux execution environments, enabling realistic testing of vectorized kernels in the presence of system calls, memory management, and runtime libraries. Additionally, QEMU integrates seamlessly with GDB, allowing interactive debugging and inspection of vector register state.

QEMU employs dynamic binary translation, which significantly reduces simulation time compared to purely interpretive simulators. This improvement in execution speed enhances development productivity when validating functional correctness across a large number of test cases and kernel configurations. Furthermore, modern QEMU versions provide architecturally accurate support for RVV 1.0 instructions, enabling reliable ISA-level validation.

\paragraph{QEMU User-Mode Emulation}

QEMU user-mode emulation is the primary execution mode used throughout this project. In this mode, RISC-V Linux binaries are executed directly on the host system without emulating a full hardware platform:

\begin{lstlisting}[language=bash]
qemu-riscv64 -cpu rv64,v=true,vlen=256 ./my_program
\end{lstlisting}

User-mode emulation offers fast startup times, direct access to the host file system, and straightforward integration with development tools. Importantly, it allows the vector length (\texttt{VLEN}) to be configured at runtime, enabling validation of Vector Length Agnostic (VLA) kernel behavior across multiple vector configurations.

\paragraph{QEMU System-Mode Emulation}

In addition to user-mode execution, QEMU system-mode emulation is employed during the development of Python wrappers for the kernel library. System-mode emulation provides a complete virtualized RISC-V machine, including bootloader, kernel, and virtual devices, enabling end-to-end testing of library integration within a full operating system environment.

This mode is particularly valuable for validating language bindings, dynamic linking behavior, and interaction between Python-based tooling and the underlying RISC-V kernel implementations. While system-mode execution incurs higher simulation overhead, it provides a realistic software stack that closely mirrors deployment scenarios.

\paragraph{Role in the Verification Workflow}

Within the overall methodology, QEMU serves as the functional validation layer. Kernel outputs produced under QEMU are compared against ONNX-based golden references to ensure correctness and portability. Once functional correctness is established, performance and microarchitectural behavior are evaluated using cycle-accurate RTL simulation on the Vicuna and Ara cores.


%%%%%%%%%%%%%%%%% ONNX SECTION %%%%%%%%%%%%%%%%%

\subsubsection{ONNX: Open Neural Network Exchange}


The Open Neural Network Exchange (ONNX) is an open standard designed to represent machine learning and deep learning models in a framework-independent manner. It was originally established through a collaboration between Facebook and Microsoft with the objective of improving interoperability across machine learning frameworks, tools, and hardware platforms.

ONNX defines a common intermediate representation for machine learning models, enabling them to be exported from one framework and executed or analyzed in another without modification. This representation is based on a computational graph abstraction, where nodes correspond to standardized operators and edges represent the flow of multi-dimensional tensors between operators.

The adoption of ONNX has been widely supported by both academia and industry, and it is now integrated into many popular machine learning frameworks and deployment toolchains. Its design emphasizes portability, reproducibility, and long-term maintainability, making it particularly suitable for systems-level research and hardware-oriented optimization efforts.

In the context of this project, ONNX serves as a unifying abstraction layer between high-level machine learning models and low-level, architecture-specific kernel implementations targeting the RISC-V Vector Extension.

\paragraph{Components of ONNX and Their Role in the Project}

\subparagraph{ONNX Model Components}

An ONNX model represents a machine learning computation as a directed computational graph, where nodes correspond to standardized operators and edges represent the flow of multi-dimensional tensor data between operators. This graph-based representation explicitly defines model inputs, outputs, and intermediate computations, providing a clear and structured description of the overall computation.

\begin{figure}[H]
    \centering
    \includegraphics[width=0.2\textwidth]{figures/ONNX-GRAPH-EX.png}
    \caption{ONNX Model Computational Graph example illustrating nodes (operators) and edges (tensor data flow).}
    \label{fig:onnx_graph}
\end{figure}

ONNX operators are drawn from a predefined and versioned operator set, with each operator having deterministic and well-specified mathematical semantics. In addition, ONNX models explicitly define tensor data types, shapes, and constant parameters, enabling consistent interpretation of computations across different software and hardware platforms. This standardized structure allows ONNX models to serve as precise and reproducible representations of intended kernel behavior.

\subparagraph{Role of ONNX in the Project}

In this project, ONNX models are used as golden references for the functional validation of machine learning kernels implemented using the RISC-V Vector Extension. The ONNX representation mirrors the functionality of the developed kernels by explicitly defining the same inputs, outputs, and computational operations, independent of any specific hardware implementation.

The RVV-based kernel outputs are compared directly against the corresponding outputs generated from ONNX models executed using a validated ONNX runtime. Since ONNX provides a standardized and hardware-agnostic format with deterministic operator behavior, it serves as a reliable baseline for correctness verification. This approach ensures that discrepancies in output can be attributed to kernel implementation issues rather than ambiguities in operator definitions or execution semantics.

By adopting ONNX as the functional reference, the project achieves reproducible and framework-independent validation, strengthening confidence that the optimized vectorized kernels preserve the intended computational behavior while improving performance.

%%%%%%%%%%%%%%%%%% RTL CORES SECTION %%%%%%%%%%%%%%%%%

\subsubsection{RTL Cores and Hardware Simulation}

Hardware simulation using Register Transfer Level (RTL) cores plays a critical role in the development and evaluation of vectorized machine learning kernels. While functional simulators are sufficient for validating instruction set compliance, they lack the temporal accuracy required to capture microarchitectural effects such as pipeline behavior, memory contention, and vector execution overheads. For this reason, this project employs cycle-accurate RTL cores to provide a realistic evaluation environment for RISC-V Vector Extension (RVV)-based kernel development.

Two RTL vector processors are used in this work: the \textit{Vicuna} vector coprocessor and the \textit{Ara} vector processor. Together, they represent complementary points in the RISC-V vector design space and provide a robust simulation foundation for functional and performance-oriented analysis.

\paragraph{Vicuna RISC-V Vector Coprocessor}

Vicuna is a 32-bit RISC-V vector coprocessor designed with a primary emphasis on timing predictability and deterministic execution. It targets the \texttt{Zve32x} subset of the RVV 1.0 specification, making it well-suited for embedded and edge-class workloads that rely on integer and fixed-point vector operations, such as quantized neural networks.

The architecture prioritizes worst-case execution time (WCET) analyzability by avoiding microarchitectural features that introduce timing variability, such as out-of-order execution or banked register files. As a result, Vicuna provides a cycle-accurate and bit-accurate simulation environment in which vector instruction latency is a deterministic function of vector length and hardware configuration. In this project, Vicuna serves as a baseline RTL platform for evaluating vector execution behavior in predictable embedded-class systems.

\paragraph{Ara Vector Processor}

Ara is a 64-bit, application-class RISC-V vector processor designed to maximize throughput and floating-point utilization for high-performance computing and machine learning workloads. It implements the full RVV 1.0 specification, supporting a wide range of data types, including IEEE-754 single- and double-precision floating-point formats.

Ara employs a lane-based vector architecture with scalable parallelism and supports advanced features such as vector chaining, masked execution, and high-bandwidth vector memory operations. Its tight integration with a scalar host core enables efficient amortization of instruction overhead across long vector sequences, making it particularly well-suited for compute-intensive kernels such as matrix multiplication, convolution, and activation functions.

In this project, Ara provides a high-fidelity RTL simulation environment for evaluating the performance and correctness of vectorized machine learning kernels under realistic architectural constraints.



\subparagraph{\textbf{A detailed architectural analysis, execution model discussion, and performance evaluation of both the Vicuna and Ara cores are presented in Section~4 of this thesis.}}


% RISC-V Vectorization Kernels Design (from presentation)
\subsection{RISC-V Vectorization Kernels Design}
This section presents the design and implementation of optimized RISC-V vector kernels for machine learning and digital signal processing applications. The kernels are organized into four fundamental patterns based on their computational characteristics and memory access behaviors. Each pattern represents a distinct class of operations commonly found in neural network inference pipelines and computer vision workloads.

% ============================================================
% PATTERN 1: Compute-Bound FMA Operations
% ============================================================
\subsubsection{Pattern 1: Compute-Bound FMA Operations}

% [EBRAHIM'S CONTENT - TO BE FILLED]

\textit{[Placeholder: ebrahim edit within this file]}

% ============================================================
% PATTERN 2: Sliding Window Kernels
% ============================================================
\subsubsection{Pattern 2: Sliding Window Kernels}

% [EBRAHIM'S CONTENT - TO BE FILLED]

\textit{[Placeholder: ebrahim edit within this file]}

% ============================================================
% PATTERN 3: Pointwise/Elementwise Kernels
% ============================================================
\subsubsection{Pattern 3: Pointwise/Elementwise Kernels}

Pointwise (elementwise) operations represent the most straightforward vectorization opportunities in machine learning workloads. These kernels exhibit several characteristics that make them ideal candidates for RISC-V vector acceleration:

\begin{itemize}
    \item \textbf{Elementwise independence}: No cross-element dependencies or data hazards
    \item \textbf{Fully contiguous memory access}: Linear, predictable access patterns
    \item \textbf{No reductions or control-flow coupling}: Minimal branching overhead
    \item \textbf{High execution frequency}: Common operations in ML inference pipelines
    \item \textbf{Immediate cost amortization}: Vector setup overhead paid back instantly
\end{itemize}

These properties make pointwise kernels the ``lowest-risk, highest-payoff'' RVV optimization targets. The kernels in this category include ReLU, Leaky ReLU, Bias Add, and Tensor Add operations.

% ------------------------------------------------------------
% ReLU and Leaky ReLU
% ------------------------------------------------------------
\paragraph{ReLU and Leaky ReLU Activation Functions}

The Rectified Linear Unit (ReLU) is one of the most widely used activation functions in modern neural networks, defined as:

\[
\text{ReLU}(x) = \max(x, 0) = 
\begin{cases}
x & \text{if } x > 0 \\
0 & \text{otherwise}
\end{cases}
\]

Leaky ReLU extends this definition to preserve small negative values using a scaling factor $\alpha$:

\[
\text{LeakyReLU}(x) = 
\begin{cases}
x & \text{if } x \geq 0 \\
\alpha \cdot x & \text{otherwise}
\end{cases}
\]

\textit{Scalar ReLU Implementation:}

\begin{lstlisting}[language=Python, caption={Scalar ReLU pseudo-code}]
for i = 0 to size-1:
    output[i] = max(input[i], 0)
end for
\end{lstlisting}

\textit{Scalar Leaky ReLU Implementation:}

\begin{lstlisting}[language=Python, caption={Scalar Leaky ReLU pseudo-code}]
for i = 0 to n-1:
    if src[i] < 0:
        dest[i] = src[i] * alpha
    else:
        dest[i] = src[i]
    end if
end for
\end{lstlisting}

\textit{Vectorized ReLU Implementation:}

The vectorized ReLU leverages vector max operations and strip-mining to process multiple elements in parallel:

\begin{lstlisting}[language=Python, caption={Vectorized ReLU pseudo-code}]
v_zero = vector_of_zeros

while there are elements left:
    vl = vector_length_for_this_iteration
    v_in = load_vector(input_pointer, vl)
    v_out = max(v_in, v_zero)
    store_vector(output_pointer, v_out, vl)
    
    advance pointers by vl
    decrease remaining_count by vl
end while
\end{lstlisting}

\textit{Vectorized Leaky ReLU Implementation:}

Leaky ReLU vectorization uses masked operations to handle the conditional behavior efficiently:

\begin{lstlisting}[language=Python, caption={Vectorized Leaky ReLU pseudo-code}]
alpha_vec = broadcast(alpha)

while remaining > largest_step (e.g., n*vector_length):
    # Load n vectors: v0, v1, ..., vn
    for each vi in {v0, v1, ..., vn}:
        negative_mask = (vi < 0)
        vi_leaky = negative_mask ? (vi * alpha_vec) : vi
        store(vi_leaky)
    end for
    
    skip forward n*vector_length elements
end while
\end{lstlisting}

The vectorized implementation processes multiple elements in parallel, with the hardware automatically handling varying vector lengths across different VLEN configurations.

% ------------------------------------------------------------
% Bias Add and Tensor Add
% ------------------------------------------------------------
\paragraph{Bias Add and Tensor Add Operations}

Bias addition is a fundamental operation in neural networks, where a learned bias vector is added to the output of convolutional or fully-connected layers. Tensor addition combines two tensors elementwise and is used throughout neural networks for residual connections and feature fusion.

\textit{Scalar Bias Add Implementation:}

\begin{lstlisting}[language=Python, caption={Scalar Bias Add pseudo-code}]
for batch in batches:
    for channel in channels:
        bias_val = bias[channel]
        for pixel in channel_pixels:
            output[...] = input[...] + bias_val
        end for
    end for
end for
\end{lstlisting}

\textit{Scalar Tensor Add Implementation:}

\begin{lstlisting}[language=Python, caption={Scalar Tensor Add pseudo-code}]
for i = 0 to size-1:
    Output[i] = A[i] + B[i]
end for
\end{lstlisting}

\textit{Vectorized Bias Add Implementation:}

The vectorized bias add broadcasts the bias value using vector-scalar operations:

\begin{lstlisting}[language=Python, caption={Vectorized Bias Add pseudo-code}]
for channel in channels:
    bias_val = bias[channel]
    
    for chunk in spatial_data by vector_size:
        vec = load_vector(input + offset)
        vec = vec + bias_val  # broadcast addition
        store_vector(output + offset, vec)
    end for
end for
\end{lstlisting}

\textit{Vectorized Tensor Add Implementation:}

Tensor addition vectorization is straightforward with vector-vector operations:

\begin{lstlisting}[language=Python, caption={Vectorized Tensor Add pseudo-code}]
set position = 0

while position < size:
    vl = min(vector_register_length, size - position)
    
    # Load vl elements from A[position...position+vl-1]
    # Load vl elements from B[position...position+vl-1]
    A_vector = load_vector(A + position, vl)
    B_vector = load_vector(B + position, vl)
    
    result = A_vector + B_vector
    
    store_vector(Output + position, result, vl)
    
    position += vl
end while
\end{lstlisting}

The vector-length agnostic programming model ensures these implementations automatically adapt to different hardware vector widths without modification.

% ============================================================
% PATTERN 4: Post-Processing Kernels (NMS)
% ============================================================
\subsubsection{Pattern 4: Post-Processing Kernels - Non-Maximum Suppression}

Non-Maximum Suppression (NMS) is a critical post-processing step in object detection pipelines that eliminates redundant overlapping bounding boxes. Unlike the previous patterns, NMS presents unique vectorization challenges:

\begin{itemize}
    \item \textbf{Post-processing characteristic}: Not compute-heavy, but memory and branch-intensive
    \item \textbf{Strong data dependencies}: Greedy sequential suppression logic
    \item \textbf{Heavy sorting and conditional logic}: Conditional suppression patterns
    \item \textbf{Irregular memory access}: Conditional operations create unpredictable patterns
    \item \textbf{Moderate vectorization gains}: Limited parallelism opportunities
\end{itemize}

However, vectorization opportunities exist in specific NMS substeps:

\begin{itemize}
    \item \textbf{Score filtering}: Vector compare and compress operations
    \item \textbf{Batched IoU computation}: Parallel min/max/sub/mul operations
    \item \textbf{Threshold comparisons}: Vector masking operations
\end{itemize}

\paragraph{Scalar NMS Algorithm}

The scalar NMS implementation follows a greedy sequential approach:

\begin{lstlisting}[language=Python, caption={Scalar NMS pseudo-code}]
# Step 1: Create list of (score, index) pairs for all detections
# Step 2: Filter: keep only pairs where score >= score_threshold
# Step 3: Sort pairs by score (DESCENDING)
# Step 4: Initialize empty list 'selected'
# Step 5: Initialize suppressed[N] = false (or use list of active candidates)

while pairs_list is not empty AND |selected| < max_output_per_class:
    a. Take top pair: current_index = pair.index
    b. Add current_index to selected
    c. current_box = boxes[current_index]
    d. For each remaining candidate j after current in sorted list:
           if suppressed[j]: continue
           candidate_box = boxes[j.index]
           if boxes do NOT overlap at all: continue
           iou = compute_iou(current_box, candidate_box)
           if iou >= iou_threshold:
               suppress j (mark as suppressed / remove from consideration)
           end if
       end for
end while

# Step 7: Return selected indices
\end{lstlisting}

\paragraph{Vectorized NMS Implementation}

The vectorized NMS strategically applies vectorization to substeps with high parallelism:

\begin{lstlisting}[language=Python, caption={Partially vectorized NMS pseudo-code}]
# Step 1: Collect high-confidence candidates (VECTORIZED)
candidates = empty list of (score, index) pairs

for i = 0 to N-1 step vector_length:
    vl = min(vector_length, N - i)
    v_scores = vector_load(scores[i:i+vl])
    mask = (v_scores >= score_thresh)
    
    if any scores pass:
        local_idx = [0 .. vl-1]
        kept_idx = compress(local_idx, mask)
        for each kept j:
            global_i = i + j
            add (scores[global_i], global_i) to candidates
        end for
    end if
end for

# Step 2: Sort candidates DESCENDING by score
# Step 3: Convert all candidate boxes to corner format
# box_corners[M][4]  (M = # of candidates after filter)

# Step 4: selected = empty list
#         suppressed = [false for all M candidates]

# Step 5: Greedy suppression with vectorized IoU
for each candidate i = 0 .. M-1 (sorted order):
    if suppressed[i]: continue
    if |selected| >= max_output_per_class: break
    
    add index of candidate i to selected
    current_box = box_corners[i]
    
    # Inner suppression (can be partially vectorized)
    for j = i+1 to M-1:
        if suppressed[j]: continue
        other_box = box_corners[j]
        
        if no overlap possible (quick reject): continue
        
        # VECTORIZED IoU computation
        iou = vectorized_iou(current_box, other_box)
        # Uses max/min/sub on (x1,y1,x2,y2)
        # -> areas, union, iou = inter/union
        
        if iou >= iou_thresh:
            suppressed[j] = true
        end if
    end for
end for
\end{lstlisting}

\paragraph{Vectorized IoU Computation}

The Intersection over Union (IoU) calculation benefits significantly from vectorization when computing IoU between one box and multiple candidate boxes simultaneously:

\begin{lstlisting}[language=Python, caption={Vectorized IoU computation pseudo-code}]
# current_box: [x1, y1, x2, y2]
# other_boxes: array of [x1, y1, x2, y2] boxes (vectorized batch)

# Broadcast current box coordinates
current_x1_vec = broadcast(current_box.x1)
current_y1_vec = broadcast(current_box.y1)
current_x2_vec = broadcast(current_box.x2)
current_y2_vec = broadcast(current_box.y2)

# Load other boxes (vector loads)
other_x1 = vector_load(other_boxes[:].x1)
other_y1 = vector_load(other_boxes[:].y1)
other_x2 = vector_load(other_boxes[:].x2)
other_y2 = vector_load(other_boxes[:].y2)

# Compute intersection coordinates (vectorized min/max)
inter_x1 = max(current_x1_vec, other_x1)
inter_y1 = max(current_y1_vec, other_y1)
inter_x2 = min(current_x2_vec, other_x2)
inter_y2 = min(current_y2_vec, other_y2)

# Compute intersection width and height (vectorized subtraction)
inter_w = inter_x2 - inter_x1
inter_h = inter_y2 - inter_y1

# Clamp to zero (no negative areas)
inter_w = max(inter_w, 0)
inter_h = max(inter_h, 0)

# Intersection area (vectorized multiplication)
inter_area = inter_w * inter_h

# Compute individual box areas
current_area = (current_box.x2 - current_box.x1) * 
               (current_box.y2 - current_box.y1)

other_w = other_x2 - other_x1
other_h = other_y2 - other_y1
other_area = other_w * other_h

# Union area = area1 + area2 - intersection
union_area = current_area + other_area - inter_area

# IoU = intersection / union (vectorized division)
iou = inter_area / union_area

return iou
\end{lstlisting}

This vectorized IoU computation processes multiple bounding box comparisons in parallel, reducing the computational overhead of the NMS algorithm's inner loop. While the overall NMS algorithm remains largely sequential due to its greedy nature, vectorizing the IoU substep provides measurable performance improvements when processing large numbers of detection candidates.

\subsubsection{Compute-Bound FMA Operations}

Neural network workloads are dominated by matrix multiplication and fully connected layers. These operations form the computational backbone of both inference and training pipelines. Their defining characteristic is extremely high arithmetic intensity: we perform massive volumes of floating-point operations relative to the amount of data moved from memory. This makes them ideal candidates for vectorization.

At the core of these operations lies the fused multiply-add (FMA), where we accumulate multiple products into a single result. Modern processors can execute FMAs in a single pipelined cycle, but scalar code wastes this potential by processing one operation at a time. Vector instructions let us pack multiple FMAs into each cycle, multiplying our computational throughput.

These kernels also exhibit regular, predictable structure. Data dependencies are minimal and well-defined, memory access patterns follow sequential or strided layouts, and control flow remains simple. This regularity plays to the strengths of SIMD architectures, where we can fill vector registers densely with minimal overhead.

\paragraph{Matrix Multiplication (GEMM)}

\subparagraph{Kernel Description}

General Matrix Multiply (GEMM) computes $C = A \times B$, where $A$ is $M \times K$, $B$ is $K \times N$, and the result $C$ is $M \times N$. Each output element represents a dot product:

\[
C[i][j] = \sum_{k=0}^{K-1} A[i][k] \cdot B[k][j]
\]

This operation underlies virtually all linear algebra in machine learning, from simple linear layers to complex attention mechanisms.

\subparagraph{Scalar Implementation}

The straightforward scalar approach uses three nested loops. For each row in $A$ and column in $B$, we accumulate a dot product by iterating through $K$ elements, multiplying corresponding pairs and summing results:

\begin{verbatim}
FOR each row i in matrix A
    FOR each column j in matrix B
        sum ← 0
        FOR k = 0 to K-1
            sum ← sum + A[i][k] × B[k][j]
        C[i][j] ← sum
\end{verbatim}

This processes one output element at a time. Each multiply-add depends on the previous accumulation, preventing any overlap of operations. The deeply nested loops add control overhead, and we completely ignore the wide vector registers sitting idle in the processor. Modern CPUs with pipelined FMA units spend most of their time stalled, waiting for data dependencies to resolve.

\subparagraph{Vectorization Strategy}

Rather than computing output elements sequentially, we can process multiple columns of the output simultaneously. Instead of accumulating into a single scalar, we maintain $\textit{vl}$ parallel accumulators in a vector register, where $\textit{vl}$ represents the number of elements that fit in one vector based on available hardware width.

This reorganization transforms the inner loop structure. For each element $A[i][k]$ in the current row, we broadcast that single value across all lanes of a vector register. We then load $\textit{vl}$ consecutive elements from row $k$ of matrix $B$ and multiply the broadcast value with this vector. The result updates all $\textit{vl}$ accumulators in parallel.

This approach delivers several advantages. We exploit the full width of vector functional units, executing $\textit{vl}$ FMAs per cycle instead of one. Memory access to matrix $B$ becomes sequential and cache-friendly, since we load contiguous elements. Loop overhead drops because we process $\textit{vl}$ outputs per iteration instead of one.

\subparagraph{Implementation}

The vectorized code replaces the innermost column loop with a while loop that processes columns in chunks. We track how many columns remain to be processed and handle them in groups of $\textit{vl}$:

\begin{verbatim}
FOR each row i in matrix A
    remaining_columns ← N
    WHILE remaining_columns > 0
        vl ← number of cols processed in parallel
        j ← N - remaining_columns
        acc_vector ← 0
        FOR k = 0 to K-1
            a_vector ← broadcast A[i][k]
            b_vector ← load B[k][j : j+vl]
            acc_vector ← acc_vector + (a_vector × b_vector)
        END FOR
        store acc_vector into C[i][j : j+vl]
        remaining_columns ← remaining_columns - vl
    END WHILE
END FOR
\end{verbatim}

For each chunk of columns, we initialize a vector accumulator to zero. The inner loop over $k$ performs the dot product computation: we broadcast each element from row $i$ of matrix $A$ across all lanes, load $\textit{vl}$ consecutive elements from the corresponding row of $B$, multiply them element-wise, and accumulate the results. After completing the dot product across all $K$ elements, we store the vector of results to the output matrix.

The broadcast operation is crucial here. It takes a single scalar value $A[i][k]$ and replicates it across every lane of a vector register, allowing that single value to be multiplied with $\textit{vl}$ different elements from $B$ simultaneously. The load from $B$ is sequential, fetching consecutive memory locations, which aligns perfectly with cache line sizes and enables efficient prefetching.

The RISC-V vector extension determines $\textit{vl}$ dynamically at runtime based on the hardware's vector register width and the element size. When the number of remaining columns isn't evenly divisible by the maximum vector length, the hardware automatically reduces $\textit{vl}$ for the final iteration, processing exactly the remaining elements without any special-case code.

\paragraph{Dense Layer (Fully Connected)}

\subparagraph{Kernel Description}

Dense layers compute weighted sums across all inputs for each output neuron. Given an input vector $x$ of size $K$ and a weight matrix $W$ of shape $N \times K$ (where each row corresponds to one output neuron), we compute:

\[
\text{output}[j] = \text{bias}[j] + \sum_{k=0}^{K-1} \text{input}[k] \cdot \text{weights}[j][k]
\]

This is essentially matrix-vector multiplication with bias addition. While conceptually similar to GEMM, we're multiplying a matrix by a single vector rather than another matrix, which affects how we structure the computation.

\subparagraph{Scalar Implementation}

The scalar code processes one output neuron at a time. We initialize each accumulator with its corresponding bias, then iterate through all input features, multiplying each by its weight and accumulating:

\begin{verbatim}
FOR each output neuron j = 0 to N-1
    acc ← bias[j]
    FOR each input feature k = 0 to K-1
        acc ← acc + input[k] × weights[j][k]
    output[j] ← acc
\end{verbatim}

By initializing with the bias value, we avoid a separate bias addition pass after computing the weighted sum. Each output neuron is computed independently with a sequential accumulation across all input features.

Like scalar GEMM, this serializes all operations. Each accumulation depends on the previous one, and we process only a single output at a time. Vector units remain completely unutilized.

\subparagraph{Vectorization Strategy}

We parallelize across output neurons, computing $\textit{vl}$ neurons simultaneously. The key insight is that for each input feature, we can broadcast that feature value and multiply it with weights from multiple neurons in parallel.

The memory access pattern differs from GEMM in an important way. In GEMM, consecutive output columns correspond to consecutive elements in memory (sequential access to $B$). Here, to compute multiple neurons in parallel, we need to load weights for the same input feature across different neurons. These weights are not contiguous in memory because the weight matrix is stored in row-major order, with each row representing one neuron's complete set of weights.

For input feature $k$, the weights we need are at positions $\text{weights}[j][k]$, $\text{weights}[j+1][k]$, $\text{weights}[j+2][k]$, etc. These addresses are separated by $K$ elements (the stride of one row), making this a strided memory access pattern rather than a sequential one.

\subparagraph{Implementation}

The vectorized implementation processes output neurons in chunks, initializing accumulators directly with bias values:

\begin{verbatim}
j ← 0
WHILE j < N
    vl ← number of outputs processed in parallel
    acc_vector ← load bias[j : j+vl]
    FOR each input feature k = 0 to K-1
        weight_vector ← load weights[j : j+vl][k]        
                                          (strided)
        input_vector ← broadcast input[k]
        acc_vector ← acc_vector + 
            (input_vector × weight_vector)
    END FOR
    store acc_vector into output[j : j+vl]
    j ← j + vl
END WHILE
\end{verbatim}

We start by loading $\textit{vl}$ bias values into the accumulator vector. This is a sequential load since bias values are stored contiguously. Then, for each input feature, we perform two key operations:

First, we load weights using strided access. The notation $\text{weights}[j:j+vl][k]$ means we're loading element $k$ from rows $j$ through $j+vl-1$. These elements are separated by $K$ positions in memory (one full row), so we use a strided load instruction with stride equal to $K \times \text{element\_size}$. The RISC-V vector ISA provides efficient strided load instructions that handle this pattern in hardware, fetching non-contiguous elements without manual gathering.

Second, we broadcast the input feature value across all vector lanes. This replicated value multiplies with the vector of weights, producing $\textit{vl}$ partial products that update the accumulators in parallel.

After processing all $K$ input features, each lane of the accumulator vector contains the complete output for one neuron (weighted sum plus bias), ready to be stored to memory.

The key advantage of initializing with bias values rather than zero is efficiency: we fold the bias addition into the initial load rather than performing a separate addition pass after the main computation. This saves both instructions and a complete pass over the output array.

\subsubsection{Sliding Window Kernels}

Sliding window operations differ fundamentally from the compute-bound kernels above. Here, a small kernel slides across a larger input, computing outputs based on local neighborhoods. These operations dominate convolutional neural networks and pooling layers.

The defining characteristic is local spatial dependency: each output depends only on a small, localized region of input determined by kernel size and stride. Consecutive output positions often use overlapping input regions, creating opportunities for data reuse. However, this also introduces irregular memory access patterns and boundary conditions that complicate vectorization.

Optimization strategies differ from dense matrix operations. While GEMM benefits from vectorizing across output dimensions, sliding window kernels often benefit more from vectorizing across output width or channels, combined with careful register blocking to exploit spatial locality.

\paragraph{2D Convolution}

\subparagraph{Kernel Description}

Two-dimensional convolution is the fundamental operation in CNNs. Given an input feature map of shape $(C_{\text{in}}, H_{\text{in}}, W_{\text{in}})$, filters with $C_{\text{out}}$ output channels, and a kernel of size $(K_h, K_w)$, we compute an output of shape $(C_{\text{out}}, H_{\text{out}}, W_{\text{out}})$.

For each output position and channel, we compute:

\[
\text{output}[oc][oh][ow] = \sum_{ic=0}^{C_{\text{in}}-1} \sum_{kh=0}^{K_h-1} \sum_{kw=0}^{K_w-1} \text{input}[ic][ih][iw] \cdot \text{kernel}[oc][ic][kh][kw]
\]

where input coordinates map to output coordinates through stride and padding:

\[
ih = oh \times \text{stride}_h - \text{pad}_h + kh, \quad iw = ow \times \text{stride}_w - \text{pad}_w + kw
\]

\subparagraph{Scalar Implementation}

The scalar approach iterates over every dimension: batch, output channel, output position, input channel, and kernel position. For each kernel element, we compute the corresponding input coordinates, verify they're within bounds (handling padding), and accumulate the product:

\begin{verbatim}
Compute output height and width
Initialize output to zero

FOR each batch b
    FOR each output channel oc, output row oh, output column ow
        sum ← 0
        FOR each input channel ic
            FOR each kernel row kh
                FOR each kernel column kw
                    ih ← oh × stride_h - pad_h + kh
                    iw ← ow × stride_w - pad_w + kw
                    IF ih, iw inside input bounds
                        sum ← sum + input[b][ic][ih][iw] 
                                  × kernel[oc][ic][kh][kw]
        output[b][oc][oh][ow] ← sum
\end{verbatim}

The output is initialized to zero at the start, then we accumulate contributions from all input channels and kernel positions. The innermost loops compute coordinate mappings and perform boundary checks to handle padding. When an input coordinate falls outside the valid range, we skip that multiply-accumulate, effectively treating out-of-bounds regions as zero (zero-padding).

This straightforward implementation processes one output element at a time. The deeply nested loops incur substantial overhead, and the boundary checks add branching to every kernel position. Worse, consecutive output positions don't systematically reuse cached data, leading to poor memory behavior.

\subparagraph{General Vectorization}

The general vectorization strategy has two phases: kernel repacking followed by parallel output channel computation. We reorganize the kernel weights so that elements for consecutive output channels become contiguous in memory.

In the original layout, kernel weights are organized as $\text{kernel}[oc][ic][kh][kw]$. To access weights for output channels $oc$, $oc+1$, $oc+2$, etc., for a fixed input channel and kernel position, we'd need to stride through memory with large jumps. By repacking into $\text{packed\_kernel}[ic][kh][kw][oc]$, we make weights for consecutive output channels adjacent in memory, enabling efficient sequential vector loads.

During computation, we process $\textit{vl}$ output channels simultaneously. For each output position, we maintain $\textit{vl}$ parallel accumulators. As we iterate through input channels and kernel positions, we broadcast each input value and multiply it with a vector of repacked weights, updating all accumulators in parallel.

\subparagraph{Implementation}

The implementation starts with kernel repacking, then executes the vectorized convolution:

\begin{verbatim}
Repack kernel so output channels are contiguous
Initialize output to zero

FOR each batch b
    FOR output channels oc in vector chunks
        FOR each output row oh, output column ow
            acc_vector ← output[b][oc:oc+vl][oh][ow]
            FOR each input channel ic
                FOR each kernel row kh
                    FOR each kernel column kw
                        ih ← oh × stride_h - pad_h + kh
                        iw ← ow × stride_w - pad_w + kw
                        IF ih, iw inside input bounds
                            input_val ← input[b][ic][ih][iw]
                            weight_vector ← packed_w[ic][kh][kw][oc:oc+vl]
                            acc_vector ← acc_vector + input_val × weight_vector
            store acc_vector to output
\end{verbatim}

We initialize the output to zero before the main computation begins. For each output position, we load the current accumulator state (which starts at zero) from the output array. This might seem redundant for the first iteration, but it allows us to accumulate contributions from multiple input channels consistently.

The key operations happen in the innermost loops. For each valid input position, we load a single scalar input value. We then load a vector of $\textit{vl}$ consecutive kernel weights from the repacked array. The scalar input value is implicitly broadcast across all lanes when multiplied with the weight vector, producing $\textit{vl}$ partial products. These accumulate into the vector register holding our parallel output channel computations.

After processing all input channels and kernel positions for a given output location, the accumulator vector contains the complete output values for $\textit{vl}$ channels at that spatial position. We store this vector back to the output array.

The repacking cost is paid once during initialization and amortized across potentially thousands of forward passes. The memory layout transformation converts strided access (which would require expensive gather operations or multiple scalar loads) into efficient sequential loads.

Boundary handling remains explicit through the conditional check. When coordinates fall outside the input bounds, we skip the multiply-accumulate entirely. This avoids the memory overhead of explicitly padding the input array, though it introduces some branching. For performance-critical applications where branches hurt, explicitly padding the input eliminates these conditionals at the cost of larger memory footprint.

\subparagraph{Convolution as Matrix Multiplication (Im2Col)}

An alternative approach transforms convolution into standard matrix multiplication. The Im2Col (image-to-column) method unfolds the input into a matrix where each column represents a flattened window. The kernel becomes a matrix where each row contains flattened weights for one output channel. Convolution then reduces to GEMM.

This transformation works for any kernel size and allows us to leverage highly optimized matrix multiplication algorithms.

\begin{verbatim}
Compute output height and width

// Step 1: Im2Col transformation
FOR each output position (oh, ow)
    FOR each input channel ic
        FOR each kernel position kh, kw
            col[K_index][oh×OW + ow] ← input value or 0 (padding)
        END FOR
    END FOR
END FOR

// Step 2: GEMM
gemm_output ← kernel_matrix × col_matrix

// Step 3: Bias Add
FOR each output channel oc
    FOR each output position
        output[oc][pos] ← gemm_output[oc][pos] + bias[oc]
    END FOR
END FOR
\end{verbatim}

The Im2Col step unfolds the input into a 2D matrix. Each column of this matrix represents one output position's receptive field: all input values that contribute to that output, flattened into a 1D vector. The row index $K\_index$ combines the input channel and kernel position indices into a single linear index.

For positions where the receptive field extends outside the input bounds (due to padding), we write zeros to the column matrix. This explicitly materializes the zero-padding in memory.

After unfolding, we have a matrix multiplication problem: $\text{kernel\_matrix}$ is $C_{\text{out}} \times (C_{\text{in}} \times K_h \times K_w)$, and $\text{col\_matrix}$ is $(C_{\text{in}} \times K_h \times K_w) \times (H_{\text{out}} \times W_{\text{out}})$. The product gives us all output values for all channels and positions.

Finally, we add bias values to each output channel. Since the GEMM produces outputs in channel-major order, we simply iterate through channels and positions, adding the appropriate bias to each element.

The advantage is that GEMM kernels are among the most optimized operations on any platform, often hand-tuned in assembly with careful cache blocking and register tiling. By reducing convolution to GEMM, we leverage this existing optimization work.

The disadvantage is memory overhead. The column matrix can be quite large: for a 224×224 input image with 64 channels and a 3×3 kernel, the unfolded matrix requires roughly 200MB of temporary storage. This overhead may be acceptable on systems with ample memory, but becomes prohibitive on embedded devices.

The transformation is particularly effective when the same input will be convolved with multiple different kernels (as in neural network layers with many output channels), since the Im2Col cost is paid once and amortized across many GEMM operations.

\subparagraph{Specialized 3×3 Vectorization}

The 3×3 kernel size deserves special attention because it dominates modern CNN architectures. When we know the kernel size is fixed at 3×3, we can apply optimizations that wouldn't be practical for variable-size kernels.

The key insight is that we can preload three consecutive input rows and keep them in registers across multiple output column computations. For a given output row, the input data from rows $oh$, $oh+1$, and $oh+2$ will be reused across all output columns in that row (assuming stride 1). By loading these rows once and reusing them, we dramatically reduce memory traffic.

\begin{verbatim}
FOR each output row oh
    row0 ← input row oh
    row1 ← input row oh+1
    row2 ← input row oh+2
    FOR output columns ow in vector chunks
        Load vectors:
            v00, v01, v02 from row0
            v10, v11, v12 from row1
            v20, v21, v22 from row2
        acc ← v00 × k00
        acc ← acc + v01 × k01
        acc ← acc + v02 × k02
        ...
        acc ← acc + v22 × k22
        store acc to output
    END FOR
END FOR
\end{verbatim}

We load three input rows into temporary storage outside the column loop. These represent the three rows of input that contribute to the current output row. Then, for each chunk of output columns, we load nine vector variables: three vectors from each of the three rows.

Each vector contains $\textit{vl}$ consecutive elements from an input row. The notation $v01$ means "vector from row 0, offset by 1 position" this captures the three horizontal positions of the kernel across multiple output columns simultaneously.

We then perform nine multiply-accumulate operations, one for each kernel position. Each operation multiplies one input vector with the corresponding kernel weight (broadcast to match the vector length) and accumulates into the result. After all nine operations, we have $\textit{vl}$ complete output values for consecutive output columns.

This approach avoids the coordinate computation overhead of the general vectorization. We don't recalculate $ih$ and $iw$ for each kernel position; instead, we directly reference preloaded vectors. The overlapping window pattern means that $v01$ for the current output column becomes $v00$ for the next, creating register reuse opportunities that a smart compiler or hand-written assembly can exploit.

Compared to Im2Col, this method uses minimal extra memory (just three row buffers) and avoids the unfolding step entirely. It's most effective for stride-1 convolutions where the input window overlap is maximal. For larger strides, the reuse benefits diminish, and Im2Col may become more attractive.

The trade-off is specificity: this approach is hard-coded for 3×3 kernels. Generalizing it to other sizes would require different loop structures and vector load patterns, whereas Im2Col works uniformly for any kernel size.

\paragraph{Max Pooling}

\subparagraph{Kernel Description}

Max pooling downsamples feature maps by taking the maximum value within each window. Given an input and a pooling kernel of size $(K_h, K_w)$, we compute:

\[
\text{output}[c][oh][ow] = \max_{kh=0}^{K_h-1} \max_{kw=0}^{K_w-1} \text{input}[c][ih][iw]
\]

where $(ih, iw)$ are determined by output position, stride, and padding. Unlike convolution, this involves only comparisons and selections, with no arithmetic operations.

\subparagraph{Scalar Implementation}

The scalar code processes one output position at a time. For each position, we compute the valid window boundaries (accounting for padding), initialize a maximum value to negative infinity, and scan all values within the window to find the maximum:

\begin{verbatim}
Compute output height and width

FOR each batch b
    FOR each channel c
        FOR each output row oh
            FOR each output column ow
                h_start ← oh × stride_h - pad_h
                w_start ← ow × stride_w - pad_w
                h_end ← min(h_start + k_h, input_height)
                w_end ← min(w_start + k_w, input_width)
                h_start ← max(h_start, 0)
                w_start ← max(w_start, 0)
                max_val ← -(inf)
                FOR h = h_start to h_end-1
                    FOR w = w_start to w_end-1
                        max_val ← max(max_val, input[b][c][h][w])
                output[b][c][oh][ow] ← max_val

\end{verbatim}

We calculate the window boundaries explicitly, clamping them to the valid input range. This handles padding by simply restricting the region we scan. We initialize the maximum to negative infinity, ensuring any actual input value will be larger.

The inner loops scan the valid window region, comparing each input value against the current maximum and updating when we find a larger value. After scanning the entire window, we write the maximum to the output.

Though conceptually simple, this scalar implementation can still bottleneck shallow networks or configurations with large stride values that reduce spatial data reuse.

\subparagraph{Vectorization Strategy}

We vectorize across output columns, computing $\textit{vl}$ output positions simultaneously. Each vector lane maintains an independent maximum accumulator. As we scan the pooling window, we load vectors of input values and compute element-wise maximums, updating all lanes in parallel.

The key insight is that maximum operations are inherently parallelizable: tracking maxima for multiple output positions requires no inter-lane communication. Each lane independently compares and updates its maximum value. The vector maximum instruction compares corresponding elements in two vectors and produces a result vector containing the element-wise maxima.

\subparagraph{Implementation}

The vectorized implementation processes output columns in chunks:

\begin{verbatim}
Compute output height and width

FOR each batch b
    FOR each channel c
        FOR each output row oh
            ih_start ← oh × stride_h - pad_h
            ow ← 0
            WHILE ow < out_w
                vl ← number of columns processed in parallel
                max_vector ← -(inf)
                FOR kh = 0 to k_h-1
                    ih ← ih_start + kh
                    IF ih out of bounds: continuez
                    FOR kw = 0 to k_w-1
                        iw ← ow × stride_w - pad_w + kw
                        input_vector ← load input values
                        max_vector ← max(max_vector, input_vector)
                store max_vector to output
                ow ← ow + vl
\end{verbatim}

We compute the starting input row position once per output row, outside the column loop. This row start is the same for all output columns in this row, so computing it once saves redundant arithmetic.

For each chunk of $\textit{vl}$ output columns, we initialize a vector of maximum values to negative infinity. This initialization broadcasts the scalar value $-\infty$ across all vector lanes, giving each lane its own independent maximum tracker.

The inner loops iterate through the pooling window. For each kernel row, we check if the corresponding input row is within bounds. If not, we skip this entire row (the \texttt{continue} statement). For each kernel column, we compute the starting input column position and load $\textit{vl}$ consecutive input values.

This load is crucial: for output column $ow$, we need input at position $ow \times \text{stride}_w - \text{pad}_w + kw$. For the next output column $ow+1$, we need input at position $(ow+1) \times \text{stride}_w - \text{pad}_w + kw$. These positions are separated by \texttt{stride\_w} in the input. So when stride is 1, we load consecutive elements; when stride is 2, we load every other element; and so on.

The RISC-V vector ISA can handle this through either strided loads (when stride more than 1) or indexed loads. The notation $\text{load input values}$ abstracts this detail, but the actual implementation would use the appropriate load instruction variant.

After loading input values, we compute the element-wise maximum with the current maximum vector. Each lane compares its input value against its current maximum and keeps whichever is larger. This process continues through all kernel positions.

Once we've scanned the entire window, the maximum vector contains the final maximum values for $\textit{vl}$ output positions. We store this vector to the output array and advance to the next chunk of columns.

The vectorization achieves significant speedup by computing $\textit{vl}$ output elements in parallel, with nearly the same work per output position as the scalar version, but amortized across $\textit{vl}$ lanes. The comparison-based nature of max pooling maps naturally to SIMD operations, making this one of the more straightforward kernels to vectorize effectively.

% Functional Verification Results
% !TEX root = ../Thesis.tex
\subsection{Functional Verification Results}

After designing and implementing vectorized kernels, the verification phase ensures functional correctness by comparing kernel outputs against trusted reference implementations. This phase ensures that vectorized implementations produce mathematically correct results, accounting for minor numerical variations inherent in floating-point arithmetic.

\subsubsection{ONNX Golden Reference Framework}

The Open Neural Network Exchange (ONNX) framework serves as the golden reference standard for functional verification. ONNX defines a hardware-agnostic computational graph representation where operations are specified as named operators and data dependencies as edges between nodes.

The verification workflow follows this sequence:

\begin{enumerate}
    \item \textbf{ONNX model creation}: A reference model is constructed in ONNX format, implementing the same computation as the RVV kernel using standard ONNX operators
    \item \textbf{Test data generation}: Identical input datasets are generated for both the RVV kernel and the ONNX model
    \item \textbf{Parallel execution}: The RVV kernel and ONNX model execute using the same inputs
    \item \textbf{Metric computation}: Output arrays are compared using quantitative metrics to account for numerical precision differences
    \item \textbf{Correctness verification}: Metrics are evaluated against predefined thresholds to confirm functional equivalence
\end{enumerate}

This approach provides several advantages:

\begin{itemize}
    \item \textbf{Hardware-agnostic validation}: ONNX references are independent of any specific CPU or accelerator
    \item \textbf{Industry standard}: ONNX is widely adopted in machine learning frameworks (TensorFlow, PyTorch, ONNX Runtime)
    \item \textbf{Reproducibility}: ONNX models can be distributed and verified independently
    \item \textbf{Compositional verification}: Complex kernels can be built from simpler verified kernels
\end{itemize}

\subsubsection{Test Data Generation Strategy}

Test datasets are carefully designed to exercise different numerical and algorithmic scenarios:

\begin{enumerate}
    \item \textbf{Boundary value testing}: Inputs include zero, small positive/negative values, large values near representable limits, and special floating-point values (NaN, infinity) where applicable
    
    \item \textbf{Random data generation}: Pseudo-random inputs drawn from uniform or normal distributions to test general-case behavior
    
    \item \textbf{Structured patterns}: Regular patterns such as identity matrices, constant arrays, and linearly-increasing sequences to facilitate manual verification and debugging
    
    \item \textbf{Real-world data samples}: Actual data from deployed models and signal processing applications to ensure practical applicability
\end{enumerate}

\subsubsection{Numerical Verification Metrics}

The framework employs two quantitative metrics to assess functional correctness:

\begin{enumerate}
    \item \textbf{Signal-to-Noise Ratio (SNR)}: Measures the ratio of the reference signal power to the error power:
    
    \begin{equation}
    \text{SNR (dB)} = 10 \cdot \log_{10}\left(\frac{\sum_i \text{ref}_i^2}{\sum_i (\text{ref}_i - \text{test}_i)^2}\right)
    \end{equation}
    
    SNR values greater than 100 dB indicate excellent agreement, with SNR of 40 dB or higher generally considered acceptable for signal processing applications.
    
    \item \textbf{Maximum Absolute Error (MaxAbs)}: Captures the largest deviation between corresponding output elements:
    
    \begin{equation}
    \text{MaxAbs} = \max_i |\text{ref}_i - \text{test}_i|
    \end{equation}
\end{enumerate}

\subsubsection{Verification Threshold Definition}

Acceptance thresholds for SNR and MaxAbs are defined based on the kernel type and numerical precision requirements:

\begin{itemize}
    \item \textbf{Element-wise operations}: SNR $>$ 100 dB, MaxAbs $<$ $10^{-6}$ for single-precision floating-point
    \item \textbf{Reduction operations}: SNR $>$ 60 dB, MaxAbs $<$ $10^{-4}$ (allowing for accumulation of rounding errors)
    \item \textbf{Complex multi-stage operations}: SNR $>$ 40 dB, MaxAbs $<$ $10^{-3}$ (accounting for multiple transformation stages)
\end{itemize}


% Discrete Functions Correctness Results (from Omar)
% !TEX root = ../Thesis.tex
\subsubsection{Discrete Functions Correctness Verification Results}

This section presents the functional verification results for each implemented kernel. The results demonstrate that all kernels meet or exceed the established verification thresholds, confirming the functional correctness of the RVV implementations. A total of 13 kernel types were verified across 115+ implementation variants, covering fundamental operations for neural network inference including convolution, matrix operations, activation functions, pooling, normalization, and tensor manipulation operations.

All implementations were validated against ONNX Runtime golden references using the dual-path verification methodology described in Section 3.4.2. Table~\ref{tab:verification_summary} provides an overview of all verified kernels, showing universal success with a 100\% pass rate. \\

\textbf{Note:} The detailed results for each kernel are available in the project repository at \url{https://github.com/OmarAly03/RaiVeX/tree/main/kernels}. Each kernel directory contains a \texttt{result.md} file with comprehensive test data.


\begin{table}[H]
\centering
\caption{Summary of kernel verification results across all implementations}
\label{tab:verification_summary}
\vspace{1em}
\begin{tabular}{l c c c }
\toprule
\textbf{Kernel Type}  & \textbf{Max Abs Error} & \textbf{SNR (dB)} & \textbf{Status} \\ 
\midrule
Conv2D                   & $6.48 \times 10^{-5}$ & 123.9 -- 139.9 & \textcolor{green}{ PASSED} \\
Conv2D Transposed       & $1.19 \times 10^{-7}$ & 143.2 -- 146.7 & \textcolor{green}{ PASSED} \\
Conv2D $3\times3$ Specialized  & 0                     & $\infty$        & \textcolor{green}{ PASSED} \\
Matrix Multiplication   & $2.86 \times 10^{-6}$ & 135.8 -- $\infty$ & \textcolor{green}{ PASSED} \\
Dense Layer              & $4.29 \times 10^{-6}$ & 132.7 -- 132.9 & \textcolor{green}{ PASSED} \\
ReLU                    & 0                     & $\infty$        & \textcolor{green}{ PASSED} \\
Leaky ReLU              & 0                     & $\infty$        & \textcolor{green}{ PASSED} \\
Max Pooling             & 0                     & $\infty$        & \textcolor{green}{ PASSED} \\
Batch Normalization     & 0                     & $\infty$        & \textcolor{green}{ PASSED} \\
Tensor Addition          & 0                     & $\infty$        & \textcolor{green}{ PASSED} \\
Bias Addition            & 0                     & $\infty$        & \textcolor{green}{ PASSED} \\
Gather Elements         & 0                     & $\infty$        & \textcolor{green}{ PASSED} \\
Scatter Elements        & 0                     & $\infty$        & \textcolor{green}{ PASSED} \\
NMS                      & 0                     & $\infty$        & \textcolor{green}{ PASSED} \\
\midrule
\textbf{}          & & & \textbf{100\% Pass Rate} \\
\bottomrule
\end{tabular}
\end{table}

\textbf{Perfect Accuracy Kernels} \\

Nine kernel types achieved perfect bitwise accuracy with zero maximum absolute error and infinite SNR across all implementation variants: ReLU, Leaky ReLU, Max Pooling, Batch Normalization, Tensor Addition, Bias Addition, Gather Elements, Scatter Elements, NMS, and the specialized Conv2D $3\times3$ kernel. This represents approximately 85\% of all tested implementations. 

These kernels achieved perfect accuracy due to their computational characteristics—either involving only elementwise operations with minimal accumulation (activation functions, additive operations), comparison-based selection (pooling), or carefully controlled normalization and indexing operations. All LMUL configurations (m1, m2, m4, m8) and optimization strategies (tiled, non-tiled) produced identical results for these kernels.

\textbf{Convolution Kernels}

Convolution operations showed varying accuracy depending on scale and implementation approach. Table~\ref{tab:conv_detailed} summarizes the results across different configurations.

\begin{table}[H]
\centering
\caption{Conv2D verification results across scales and implementations}
\label{tab:conv_detailed}
\vspace{1em}
\begin{tabular}{l l c c}
\toprule
\textbf{Configuration} & \textbf{Implementation} & \textbf{Max Abs Error} & \textbf{SNR (dB)} \\ 
\midrule
\multirow{2}{*}{Small ($8\times8$)}     
                & All RVV variants        & $7.15 \times 10^{-7}$ & 139.9 \\
                & Scalar baseline         & $9.54 \times 10^{-7}$ & 139.0 \\
\midrule
\multirow{3}{*}{Medium ($26\times26$)}
                & All RVV variants        & $3.81 \times 10^{-5}$ & 127.6 \\
                & IM2COL+GEMM            & $6.29 \times 10^{-5}$ & 124.0 \\
                & Scalar baseline         & $6.48 \times 10^{-5}$ & 123.9 \\
\midrule
\multirow{1}{*}{$3\times3$ Specialized ($128\times128$)}
                & All variants  & 0                     & $\infty$ \\
\bottomrule
\end{tabular}
\end{table}

All RVV vectorized implementations (m1 through m8) achieved identical accuracy within each configuration, demonstrating LMUL-independent numerical consistency. Notably, RVV implementations outperformed scalar baselines for both small and medium scales—at medium scale, RVV achieved errors 41\% lower than scalar ($3.81 \times 10^{-5}$ vs $6.48 \times 10^{-5}$) with 3.7 dB better SNR. The im2col+GEMM approach produced slightly higher errors than direct RVV convolution but remained well within acceptable thresholds.

The specialized $3\times3$ kernel achieved perfect accuracy across all eight variants (m1/m2/m4/m8 in both batched and non-batched configurations), demonstrating the benefits of kernel-specific optimization.


\textbf{Transposed Convolution} \\

Transposed convolution showed excellent accuracy across three different input scales, with implementation performance varying by problem size. Table~\ref{tab:conv_transposed} summarizes the results.

\begin{table}[H]
\centering
\caption{Transposed convolution verification across scales ($3\times3$ kernel, stride=1)}
\label{tab:conv_transposed}
\vspace{1em}
\begin{tabular}{l l c c}
\toprule
\textbf{Scale} & \textbf{Implementation Type} & \textbf{Max Abs Error} & \textbf{SNR (dB)} \\ 
\midrule
\multirow{3}{*}{Small ($4\times4$)} 
    & RVV $3\times3$ Specialized (m1--m8)  & $6.71 \times 10^{-8}$ & 146.7 \\
    & RVV General Purpose (m1--m8)  & $1.19 \times 10^{-7}$ & 143.9 \\
    & Scalar Baseline               & $1.19 \times 10^{-7}$ & 143.2 \\
\midrule
\multirow{3}{*}{Medium ($26\times26$)} 
    & RVV General Purpose (m1--m8)  & $4.77 \times 10^{-7}$ & 141.6 \\
    & RVV $3\times3$ Specialized (m1--m8)  & $4.77 \times 10^{-7}$ & 140.9--141.2 \\
    & Scalar Baseline               & $4.77 \times 10^{-7}$ & 141.3 \\
\midrule
\multirow{3}{*}{Large ($128\times128$)} 
    & RVV General Purpose (m1--m8)  & $7.15 \times 10^{-7}$ & 141.4 \\
    & RVV $3\times3$ Specialized (m1--m8)  & $4.77 \times 10^{-7}$ & 141.2--141.3 \\
    & Scalar Baseline               & $4.77 \times 10^{-7}$ & 141.6 \\
\bottomrule
\end{tabular}
\end{table}

At small scale ($4\times4$ input), the specialized $3\times3$ kernel achieved 43\% lower error than both general RVV and scalar implementations ($6.71 \times 10^{-8}$ vs $1.19 \times 10^{-7}$) with 2.8--3.5 dB better SNR. However, this advantage diminishes at larger scales.

For medium and large scales ($26\times26$ and $128\times128$), all implementations converged to similar accuracy levels, with maximum absolute errors ranging from $4.77 \times 10^{-7}$ to $7.15 \times 10^{-7}$ and SNR values between 140.9--141.6 dB. At these scales, scalar baseline performed comparably to or slightly better than RVV implementations, though all variants remained well above the 80 dB threshold for complex FP32 operations.

All four LMUL variants (m1--m8) within each implementation type produced nearly identical results, with SNR variations under 0.5 dB at medium/large scales—indicating LMUL choice affects performance but not accuracy.

\textbf{Matrix Multiplication and Dense Layer} \\

Matrix multiplication demonstrated remarkable numerical properties, with RVV implementations achieving superior accuracy to scalar baselines.

\begin{table}[H]
\centering
\caption{Matrix multiplication verification ($64\times64$)}
\label{tab:matmul_detailed}
\vspace{1em}
\begin{tabular}{l c c}
\toprule
\textbf{Implementation Type} & \textbf{Max Abs Error} & \textbf{SNR (dB)} \\ 
\midrule
All RVV Variants & 0 & $\infty$ \\
Scalar Baseline   & $1.91 \times 10^{-6}$ & 139.0 \\
Tiled Scalar  & $2.86 \times 10^{-6}$ & 135.8 \\
\bottomrule
\end{tabular}
\end{table}

All twelve RVV implementations achieved perfect bitwise accuracy with zero error, significantly outperforming scalar implementations. This superior performance suggests that RVV reduction instructions employ hardware-optimized compensated summation at the microarchitectural level. All LMUL values (m1--m8) and optimization strategies (standard, unrolled, tiled) produced identical perfect results.

Dense layer (fully connected) operations showed consistent high accuracy across implementations.

\begin{table}[H]
\centering
\caption{Dense layer verification (batch=1, $128\times128$)}
\label{tab:dense_detailed}
\vspace{1em}
\begin{tabular}{l c c}
\toprule
\textbf{Implementation Type} & \textbf{Max Abs Error} & \textbf{SNR (dB)} \\ 
\midrule
All RVV Variants (m1--m8, 4 total) & $4.29 \times 10^{-6}$ & 132.9 \\
Scalar Baseline                     & $2.86 \times 10^{-6}$ & 132.7 \\
\bottomrule
\end{tabular}
\end{table}

Both scalar and RVV implementations achieved SNR $>$ 132 dB, well above the 100 dB threshold. All LMUL variants produced identical results, with RVV showing slightly higher error than scalar but maintaining excellent accuracy for this matrix-vector multiplication pattern.


\textbf{Numerical Accuracy Analysis} \\

The verification results demonstrate exceptional numerical accuracy across all RVV-vectorized kernels, significantly exceeding the established thresholds. Out of 115+ implementation variants tested across 13 kernel types, the observed performance can be categorized into three accuracy tiers:

\paragraph{Perfect Accuracy (Tier 1):} The majority of kernels---including all activation functions (ReLU, Leaky ReLU), pooling operations (MaxPool), normalization (BatchNorm), tensor manipulation operations (addition, bias addition, gather, scatter), and NMS---achieved perfect bitwise accuracy with zero maximum absolute error and infinite SNR. This represents approximately 85\% of all tested implementations. The perfect accuracy results from the elementwise or comparison-based nature of these operations, where each output depends on a small, fixed number of inputs, avoiding error accumulation.

\paragraph{Excellent Accuracy (Tier 2):} Matrix multiplication achieved zero error for all RVV implementations while scalar implementations showed errors up to $2.86 \times 10^{-6}$ (SNR = 135.8 dB). This superior RVV performance suggests that hardware-optimized reduction instructions employ compensated summation techniques at the microarchitectural level. Dense layers achieved SNR values of 132.7--132.9 dB with errors on the order of $10^{-6}$. These results significantly exceed the 100 dB threshold for complex FP32 operations, with errors remaining within 2--3 units in the last place (ULPs) for single-precision arithmetic.

\paragraph{Very High Accuracy (Tier 3):} Convolution operations, the most computationally complex kernels, achieved SNR values between 123.9 and 146.7 dB. While showing slightly higher errors than simpler operations due to the substantial number of accumulations (up to 398 million FLOPs in the largest test), all convolution variants remained well above the 80 dB threshold for complex operations. The im2col+GEMM approach produced slightly higher errors than direct convolution but remained well within acceptable bounds; whereas the specialized $3\times3$ convolution implementation achieved perfect accuracy, demonstrating the effectiveness of kernel-specific optimizations.

A critical finding is the \textbf{LMUL-independent accuracy}: all LMUL configurations (m1, m2, m4, m8) produced numerically identical results within each kernel. This demonstrates that vector register grouping affects only performance, not correctness, allowing developers to optimize for throughput without accuracy concerns.

% what is this textbf trying to add? is this placement appropriate?
% \textbf{Comparison with ONNX Runtime}

% The selection of ONNX Runtime as the golden reference was based on several considerations. ONNX Runtime is widely recognized as a high-performance, production-grade inference engine that implements the ONNX specification accurately and is extensively tested across various platforms, hardware accelerators (CPU, GPU, NPU), and deployment scenarios. It serves as a de facto standard for cross-platform ML inference. By comparing against ONNX Runtime, we ensure that our implementations conform to industry-standard ML operator semantics and produce results compatible with models trained using popular frameworks (PyTorch, TensorFlow, etc.).


% Models (from presentation)
% !TEX root = ../Thesis.tex
\subsubsection{Models}

\textit{[Placeholder]}


This chapter has detailed the end-to-end methodology for developing the RaiVeX Library. It motivated the selection of representative ML kernels, described the RISC-V toolchain, QEMU-based execution environment, and ONNX-based verification framework, and presented the RVV vectorization strategies applied across compute-bound, sliding-window, pointwise, indexing, normalization, and post-processing kernels. Functional verification at both discrete-kernel and full-model levels confirmed numerical correctness and composability, establishing a solid foundation for the performance analysis presented in the next chapter.


% =============================================================================
% Methodology: Performance Validation
% =============================================================================
\newpage
\section{Performance Analysis}

This chapter evaluates the performance of the RVV-accelerated kernels implemented in the RaiVeX Library. It first introduces the RTL-based hardware platforms and simulation environment used for benchmarking, then details the validation strategy for measuring cycle-accurate execution time on the Ara vector coprocessor. The chapter then presents quantitative performance results for representative compute-bound, memory-bound, and pointwise kernels, highlighting the impact of vector length configuration, tiling, and vectorization strategies on overall speedup.

% Hardware (RTL Cores) (from hardware.tex)
% !TEX root = ../Thesis.tex
\subsection{Hardware (RTL Cores)}

In the rigorous domain of computer architecture research, particularly within the context of next-generation machine learning (ML) workload acceleration, the simulation environment serves as the foundational bedrock for all performance claims and design space explorations. While high-level functional simulators—such as Spike or QEMU—provide a mechanism for validating instruction set architecture (ISA) compliance and functional correctness, they fundamentally lack the temporal fidelity required to model complex microarchitectural phenomena.

\subsubsection{Role of RTL Cores in Architectural Research}

For a graduation thesis focused on the benchmarking of RISC-V vector architectures, relying solely on functional simulation would obscure critical bottlenecks such as pipeline hazards, register file banking conflicts, memory interconnect contention, and the latency costs associated with control flow divergence. Register Transfer Level (RTL) cores, therefore, play an indispensable role. They offer a bit-accurate and cycle-accurate representation of the hardware, synthesized from languages such as System Verilog.

Simulation at this level allows the researcher to observe the precise interaction between the scalar host processor and the vector accelerator, capturing the ``handshake'' overheads that are often idealized in abstract models. Furthermore, RTL simulation is the only methodology capable of generating credible Power, Performance, and Area (PPA) metrics. By simulating the actual hardware description that would eventually be mapped to silicon or Field-Programmable Gate Arrays (FPGAs), researchers can derive energy efficiency numbers (e.g., FLOPS/Watt) and area utilization statistics (e.g., gate counts or LUT usage) that are grounded in physical reality rather than theoretical estimation.

For machine learning workloads, which are characteristically defined by dense linear algebra operations (GEMM), convolutions (CONV2D), and high-bandwidth memory access patterns, RTL cores reveal the true utilization of functional units (FUs). They allow for the precise measurement of ``raw throughput ideality''—a metric comparing achieved performance against theoretical peaks—and facilitate the identification of non-obvious bottlenecks, such as the scalar core's instruction issue rate limiting the performance of short-vector kernels.

\subsubsection{Importance of Cycle-Accurate Simulation}

The evaluation of vectorized ML kernels requires a simulation environment that can faithfully model the behavior of the RISC-V Vector (RVV) extension. The RVV specification introduces a paradigm of data-level parallelism that is significantly more complex than traditional SIMD (Single Instruction, Multiple Data) approaches found in fixed-width architectures. Features such as Vector Length Agnosticism (VLA), dynamic Element Width (SEW) grouping (LMUL), and masked execution create a vast design space where theoretical efficiency does not always translate to realized performance.

Cycle-accurate simulation is paramount for evaluating these kernels because it exposes the latency penalties associated with microarchitectural housekeeping. For instance, the ``strip-mining'' loops common in ML kernels require the hardware to dynamically adjust the vector length (\texttt{vsetvli}) and handle potentially misaligned memory accesses. An RTL simulation reveals the setup time of the vector pipeline, the latency of the Vector Load/Store Unit (VLSU) when handling strided accesses (common in tensor operations), and the impact of coherent cache hierarchies on memory bandwidth.

Without cycle-accurate visibility, a researcher might overestimate the performance of a matrix multiplication kernel by failing to account for the cycles lost to cache invalidations or the serialization of micro-operations within the vector unit. Moreover, ML workloads often exhibit phases of computation that are distinct: memory-bound phases (e.g., activation loading) and compute-bound phases (e.g., matrix accumulation). RTL cores allow for the construction of ``Roofline'' models based on empirical data, plotting arithmetic intensity against achieved floating-point operations per second.

\subsubsection{Rationale for Selecting Ara and Vicuna}

To provide a comprehensive evaluation of the RISC-V vector design space, this research employs two distinct RTL cores: Ara and Vicuna. These cores were selected because they represent two divergent philosophies within the architectural spectrum: high-performance computing (HPC) and predictable real-time embedded systems.

\textbf{Ara} is selected as the representative for high-performance, throughput-oriented vector processing. As the first fully open-source implementation of the ratified RVV 1.0 specification, Ara targets application-class workloads. It is designed to scale to high lane counts (up to 16 lanes) and focuses on maximizing floating-point utilization for complex kernels like those found in training and heavy inference. Its inclusion enables the benchmarking of ``scale-up'' scenarios where raw computational power and energy efficiency are the primary metrics.

\textbf{Vicuna}, in contrast, is selected to represent the safety-critical and embedded domain. It implements the Zve32x subset of the vector extension (integer only) and prioritizes timing predictability over maximum average-case throughput. Vicuna's design ensures freedom from timing anomalies, making it a unique platform for studying how vectorization can be applied in real-time systems where Worst-Case Execution Time (WCET) bounds are mandatory. Its inclusion allows the research to explore the trade-offs between determinism and performance, a critical consideration for ML deployment in autonomous vehicles and industrial control systems.

By juxtaposing these two cores, the benchmarking environment covers the full breadth of the RISC-V vector ecosystem, from the cloud (Ara) to the edge (Vicuna), providing a nuanced and exhaustive analysis of hardware RTL cores for ML workloads.

% ------------------------------------------------------------
% Ara Vector Processor
% ------------------------------------------------------------

\subsection{Ara Vector Processor}

\subsubsection{Overview and Design Motivation}

Ara is a 64-bit vector unit (VPU) designed to act as a high-performance coprocessor for the CVA6 (formerly Ariane) scalar core. It is engineered specifically to address the performance requirements of data-parallel workloads found in High-Performance Computing (HPC) and Artificial Intelligence (AI). The architecture is rooted in the historical lineage of vector processing—tracing its conceptual origins to the Cray-1 supercomputer—and aims to mitigate the Von Neumann Bottleneck. By leveraging the Single Instruction, Multiple Data (SIMD) execution model, Ara amortizes the high energy and latency costs of instruction fetching and decoding over large vectors of data, thereby boosting both performance and energy efficiency.

The primary target workloads for Ara are those exhibiting high degrees of data-level parallelism, such as dense linear algebra (e.g., \texttt{dgemm}, \texttt{sgemm}), convolutions for Deep Neural Networks (DNNs), and scientific computing kernels like Fast Fourier Transforms (FFT). The design motivation explicitly references the Fugaku supercomputer's A64FX processor as a contemporary proof point for the viability of vector architectures in the exascale era, positioning Ara as a RISC-V counterpart aiming for similar efficiency in the open-source hardware domain.

Ara's architectural evolution (specifically the transition from Ara to Ara2) was driven by two overriding goals: strict compliance with the ratified RISC-V Vector Extension version 1.0 (RVV 1.0) and the maximization of floating-point throughput. The transition to RVV 1.0 necessitated significant microarchitectural changes to support new behaviors for masking, element width handling, and vector configuration. The throughput scaling goal enables Ara to support a parametric number of lanes ranging from 2 to 16, allowing the processor to be instantiated in various PPA configurations while achieving high functional unit utilization (targeting $>90\%$ on compute-bound kernels).

\subsubsection{Architectural Organization}

The Ara system operates as a coherent coprocessor system. The scalar host core, CVA6, is responsible for the control plane: it fetches instructions, handles interrupts, and performs scalar computations. When CVA6 encounters a vector instruction, it offloads the instruction to Ara via a dedicated accelerator interface. This interface is designed to be non-speculative, meaning instructions are dispatched only when they are committed by the scalar core, simplifying the vector unit's control logic by removing the need for complex rollback mechanisms in the event of branch mispredictions.

The system shares a unified memory hierarchy. Both CVA6 and Ara access main memory via an AXI interconnect. A critical component of this organization is the enforcement of memory consistency between the scalar and vector domains without requiring software-managed coherence (explicit fences), a significant architectural enhancement over previous iterations.

\begin{figure}[H]
\centering
\includegraphics[width=\textwidth]{figures/ara.jpg}
\caption{Top-level block diagram of the Ara2 system showing the vector coprocessor, detailed lane architecture, and host scalar core CVA6 integration.}
\label{fig:ara_diagram}
\end{figure}

Ara employs a scalable, lane-based architecture. The computational resources and the Vector Register File (VRF) are distributed horizontally across $L$ lanes. Each lane acts as an independent slice of the vector machine, containing a portion of the VRF and dedicated functional units. In a 16-lane configuration, the processor can effectively compute 16 double-precision (64-bit) operations per clock cycle.

Within each lane, Ara instantiates specific execution units: a Vector ALU (VALU) for integer arithmetic and logical operations, a Vector FPU (VFPU) for floating-point arithmetic (FMA, Add, Mul) with native double-precision (FP64) support, and a Vector Multiplier (VMUL) specialized for integer multiplication. Ara utilizes a standard lane-striped layout where consecutive elements of a vector are placed in consecutive lanes (e.g., Element 0 in Lane 0, Element 1 in Lane 1).

\subsubsection{RVV Implementation}

Ara fully implements the RVV 1.0 frozen extension with comprehensive feature support:

\begin{itemize}
    \item \textbf{Data Types:} IEEE-754 floating-point (FP16, FP32, FP64) and standard integers (INT8, INT16, INT32, INT64)
    \item \textbf{Reductions:} Full support for vector reduction operations, including ordered and unordered floating-point reductions requiring careful pipeline management
    \item \textbf{Masking:} Comprehensive support for masked execution where operations on specific elements can be suppressed based on a mask register
    \item \textbf{Permutations:} Instructions such as \texttt{vslideup}, \texttt{vslidedown}, and \texttt{vgather}/\texttt{vscatter} supported through on-chip interconnect
\end{itemize}

\subsubsection{Vector Execution Model}

Ara adheres to the Vector Length Agnostic (VLA) programming model mandated by the RISC-V specification. The hardware parameter VLEN (bits per vector register) can vary between instantiations, but the software binary remains compatible. At runtime, the \texttt{vsetvli} instruction configures the application vector length (AVL). Ara's control logic automatically stripes this AVL across the available lanes. If the requested vector length exceeds the physical capacity of the parallel lanes ($VL > Lanes$), the hardware ``strip-mines'' the operation in hardware, executing the vector in temporal chunks.

A defining complexity of RVV 1.0 is the support for dynamic Single Element Width (SEW) changes. Because Ara maps consecutive elements to consecutive lanes, changing the SEW changes which lane holds a specific logical element. When an instruction uses source operands with different SEWs (mixed-width operations), Ara injects reshuffle micro-operations. Before the main arithmetic operation executes, the Slide Unit (SLDU) is engaged to realign the data bytes across the lanes to match the target SEW.

The execution pipeline is decoupled. Once an instruction is dispatched from CVA6 to Ara's Dispatcher, it enters an instruction queue (Issue Queue). Instructions are issued to the functional units when operands are available and execute in a SIMD manner. Ara supports vector chaining, allowing a dependent instruction to begin execution before the predecessor has fully completed, provided the necessary elements are available. This is essential for keeping the deep pipelines of the FPU utilized during complex sequences like \texttt{vfmacc} (fused multiply-accumulate).

\subsubsection{Memory Subsystem}

The Vector Load/Store Unit (VLSU) is the interface between the high-bandwidth AXI memory system and the parallel vector lanes. Its primary responsibility is to fetch data from memory and align it to the correct lanes in the VRF. The VLSU is one of the most complex units in the design, with area and complexity scaling superlinearly with the number of lanes ($O(L^2)$) because it must route data from a fixed-width memory bus to any of the lanes depending on the stride and element index. It handles unit-stride loads (contiguous blocks), strided loads (regular gaps), and indexed loads (scatter/gather).

Ara2 introduces a robust hardware coherency scheme. The CVA6 data cache (L1-D) is configured in write-through mode, ensuring that any data written by the scalar core is immediately visible in main memory where Ara reads its data. When Ara performs a vector store, it sends invalidation signals to CVA6, forcing it to invalidate lines corresponding to the addresses written by the vector unit. This hardware mechanism eliminates the need for fence instructions to maintain coherence between scalar and vector memory views.

\subsubsection{RTL Implementation}

Ara is implemented in System Verilog and is designed to be technology-agnostic, though it is optimized for modern ASIC nodes. The reference implementation is characterized in 22nm FD-SOI technology, achieving a target frequency of 1.35 GHz for configurations up to 8 lanes. The critical path is approximately 40 FO4 (Fan-Out-of-4) inverter delays, indicating a moderately aggressive pipeline depth suitable for high-performance operation.

The design is highly parameterized with lane counts of 2, 4, 8, or 16. As lane count increases, the area of the functional units scales linearly. However, the area of the interconnects—specifically the Slide Unit (SLDU) and the Mask Unit (MASKU)—scales superlinearly. Ara2 implements a specific optimization restricting the SLDU to power-of-two strides, reducing the wiring complexity from $O(L^2)$ to $O(L \log L)$ and enabling feasibility at 16 lanes.

\subsubsection{Benchmarking Suitability}

Ara is exceptionally well-suited for benchmarking compute-bound ML kernels. For large matrices (e.g., $128 \times 128$), Ara achieves 97-99\% FPU utilization, indicating that the microarchitecture successfully hides memory latency and control overhead. The bit-accurate nature of Ara allows researchers writing kernels using RVV intrinsics to precisely tune their code, verifying lane utilization and optimizing register allocation. Research highlights that for smaller problem sizes, multi-core configurations with smaller vector units can outperform single large units, providing critical insights for architectural trade-offs.

% ------------------------------------------------------------
% Vicuna Vector Processor
% ------------------------------------------------------------

\subsection{Vicuna RISC-V Vector Coprocessor}

\subsubsection{Overview and Design Motivation}

Vicuna is a 32-bit vector coprocessor designed to fill a distinct niche in the RISC-V ecosystem: timing predictability. While most vector processors maximize average-case throughput using caches, out-of-order execution, and banking, these features introduce ``timing anomalies''—situations where a local speedup results in a global slowdown due to pipeline scheduling effects. Vicuna's primary purpose is to serve real-time systems (e.g., automotive ADAS, avionics) where the Worst-Case Execution Time (WCET) must be strictly bounded and analyzable.

Despite its focus on predictability, Vicuna does not sacrifice scalability. It is designed to scale its performance linearly with the number of execution units while maintaining a simple, analyzable timing model. It specifically targets the Zve32x extension—a subset of RVV 1.0 intended for embedded processors that require vectorization for integer workloads (like quantized neural networks) but do not need 64-bit elements or floating-point support.

\subsubsection{Architectural Organization}

Vicuna acts as a coprocessor to a main scalar core. The reference integration uses the Ibex core (a small, efficient 2-stage RISC-V core) or the CV32E40X. Communication is handled via the OpenHW Group's CORE-V eXtension Interface (XIF), where the main core fetches instructions and dispatches valid vector instructions to Vicuna.

Vicuna is highly configurable and supports different internal pipeline organizations. In the \textbf{Compact} configuration, all functional units share a single pipeline. In the \textbf{Dual/Triple} configuration, units are distributed across multiple parallel pipelines (e.g., Memory Unit in Pipeline A, ALU in Pipeline B), allowing for concurrent execution of loads and arithmetic while improving efficiency without introducing unpredictable hazards.

\begin{figure}[H]
\centering
\includegraphics[width=\textwidth]{figures/Vicuna.jpg}
\caption{Overview of Vicuna's architecture and its integration with the Ibex main core. Both cores share a common data cache with predictable memory arbitration ensuring deterministic timing behavior.}
\label{fig:vicuna_diagram}
\end{figure}

Vicuna executes vector instructions using a dedicated set of functional units: a Vector Load/Store Unit (VLSU) for memory traffic, a Vector ALU (VALU) for integer arithmetic and logic, a Vector Multiplier (VMUL) for integer multiplication, and Vector Slide (VSLD) and Element (VELEM) units for permutations and reductions. The control logic is designed to be monotonic, ensuring that the progress of an instruction is never hindered by a subsequent instruction—a key requirement for preventing timing anomalies.

\subsubsection{RVV Implementation}

Vicuna implements the RVV 1.0 (Zve32x) extension profile with support for 8-bit, 16-bit, and 32-bit integers. It explicitly excludes floating-point operations and 64-bit element support, which reduces area and complexity while aligning with its embedded target. Vicuna supports configurable vector register lengths (VLEN), typically synthesized with 512-bit sizes in FPGA tests, and handles Element Widths (SEW) of 8, 16, and 32 bits. The execution model ensures that the processing time for a vector of length $N$ is a deterministic function of $N$ and the number of execution units, enabling precise WCET calculation.

\subsubsection{Execution Model}

Vicuna strictly adheres to in-order execution. Instructions are issued to the functional units only when all dependencies are resolved. The pipeline is designed such that a local timing variation (e.g., a stall) never induces a larger global timing variation, allowing for compositional timing analysis where the timing of the vector unit can be analyzed independently of the scalar core's complex state.

Parallelism in Vicuna is achieved through simultaneous and successive processing. Multiple elements are processed in a single cycle if the data path width allows (e.g., processing four 8-bit elements on a 32-bit datapath). For vectors longer than the datapath width, the unit processes chunks sequentially over multiple cycles, amortizing the instruction fetch cost through temporal vectorization.

\subsubsection{Memory Subsystem}

Vicuna supports the standard RVV memory access patterns: unit-stride, strided, and indexed (scatter/gather). To maintain predictability, the handling of these accesses is strictly serialized or arbitrated in a fixed manner. Vicuna employs a specialized memory arbiter to ensure that vector memory accesses do not cause counter-intuitive interference with the scalar core. If a vector load is active, the scalar core might stall, but the duration of this stall is bounded and predictable, allowing system architects to calculate worst-case scenarios for critical interrupts.

\subsubsection{RTL Implementation}

Vicuna is implemented in SystemVerilog with a focus on FPGA deployment, particularly the Xilinx 7 Series. The design is compact, with resource utilization (LUTs and Flip-Flops) comparable to other soft-core vector processors like VESPA or VEGAS, yet offering higher performance due to its pipelining and RVV compliance. On a Xilinx 7 Series FPGA, Vicuna achieves a clock frequency of 80 MHz with a peak throughput of 10.24 billion operations per second for 8-bit operations (128 MACs/cycle).

The verification strategy for Vicuna focuses on proving timing constancy using Verilator, Questasim, and xsim. The primary verification metric is that benchmarks (e.g., matmul) must execute in the exact same number of cycles for every run, regardless of input data values. This confirms the absence of timing anomalies through repeated validation using a suite of benchmarks (AXPY, CONV2D, GEMM).

\subsubsection{Benchmarking Suitability}

Vicuna serves as the baseline for embedded efficiency in the thesis benchmarking suite, representing the ``constrained'' design point. Benchmarking on Vicuna allows for the construction of Roofline models for embedded devices, demonstrating that for compute-bound kernels like GEMM, Vicuna achieves $>90\%$ efficiency. Since Vicuna focuses on integer arithmetic, it is ideal for evaluating 8-bit quantized neural networks, providing direct insight into how RISC-V vectors can accelerate edge AI applications without the power and area cost of floating-point hardware.

% ------------------------------------------------------------
% Comparative Analysis
% ------------------------------------------------------------

\subsection{Comparative Analysis: Ara vs. Vicuna}

The most fundamental difference between Ara and Vicuna lies in their architectural philosophy: Ara is an Application-Class Processor targeting maximum throughput, while Vicuna is an Embedded-Class Coprocessor prioritizing timing predictability.

\begin{table}[H]
\centering
\caption{Architectural Comparison of Ara and Vicuna}
\begin{tabular}{p{3.5cm} p{5cm} p{5cm}}
\toprule
\textbf{Feature} & \textbf{Ara (Ara2)} & \textbf{Vicuna} \\
\midrule
ISA Compliance & RVV 1.0 (Full, incl. FP64) & Zve32x (Integer Only) \\
Host Core & CVA6 (Linux-capable, 6-stage) & Ibex (RTOS-capable, 2-stage) \\
Primary Goal & Maximize Throughput & Maximize Predictability (WCET) \\
Scaling Mechanism & Lane Replication (2-16) & Pipeline Parallelism \\
Implementation & ASIC (22nm FD-SOI) & FPGA (Xilinx 7 Series) \\
Data Path & 64-bit (Double Precision) & 32-bit (Integer) \\
Target Frequency & 1.35 GHz & 80 MHz \\
Peak Performance & 97-99\% FPU utilization & $>90\%$ efficiency \\
Memory Coherency & Hardware coherent & Predictable arbitration \\
\bottomrule
\end{tabular}
\label{tab:ara_vicuna_comparison}
\end{table}

\textbf{Execution Model:} Ara's execution model allows for dynamic optimization through chaining, out-of-order writebacks (within scoreboard bounds), and complex reshuffling, maximizing utilization but making exact cycle-count prediction difficult. Vicuna's in-order, monotonic model guarantees that if a task takes N cycles once, it will always take N cycles. Ara uses a scoreboard and hazard detection logic to stall the pipeline dynamically, while Vicuna relies on a stricter structural hazard resolution strategy that prevents hazards by construction.

\textbf{Memory Subsystem:} Ara employs a hardware-coherent memory system where interactions between the CVA6 write-through cache and the vector unit are managed automatically, enabling seamless shared-memory programming. Vicuna uses a predictable memory arbiter with strict access ordering to ensure the vector and scalar units do not interfere in unpredictable ways, simplifying hardware at the cost of more careful software management.

\textbf{Benchmarking Trade-offs:} Ara excels at floating-point benchmarks (\texttt{dgemm}, \texttt{sgemm}), pushing the limits of RVV 1.0 in terms of raw FLOPs, but incurs high simulation cost and complexity due to superlinear scaling of interconnects. Vicuna serves as the definitive reference for hard real-time vectorization, proving vectors are safe for safety-critical systems, with lightweight and fast simulation, but is limited to integer workloads and cannot benchmark floating-point training kernels.

This duality ensures that the benchmarking results are robust, covering the diverse requirements of the modern computing spectrum from cloud (Ara) to edge (Vicuna) deployments.


% Validation Strategy (from Omar)
% !TEX root = ../Thesis.tex
\subsection{Validation Strategy}

\textit{[Placeholder]}


% Validation Results (from Omar)
% !TEX root = ../Thesis.tex
\subsection{Validation Results}

\textit{[Placeholder]}


This chapter has presented a cycle-accurate performance evaluation of the RaiVeX Library on the Ara vector coprocessor. By comparing RVV-based kernels against scalar baselines, we quantified speedups across matrix multiplication, convolution, pooling, activation, normalization, and additive operations, and analyzed the sensitivity of performance to LMUL, problem size, and data layout. The results demonstrate that, for sufficiently large and compute-intensive workloads, RVV 1.0 on Ara sustains substantial acceleration—often exceeding an order of magnitude—while memory-bound kernels approach a bandwidth-limited regime around $20\times$ speedup. These findings confirm that the methodology and kernel designs introduced in the previous chapter translate into practical throughput gains on realistic RISC-V vector hardware, supporting the case for RVV-based acceleration of embedded ML inference.

% =============================================================================
% Open Source Library Architecture
% =============================================================================
\newpage
% !TEX root = ../Thesis.tex
\section{Open Source Library Architecture}

\textit{[Placeholder]}


% =============================================================================
% Conclusion & Future Work
% =============================================================================
\newpage
\section{Conclusion and Future Work}

\subsection{Conclusion}

This thesis presented the design, implementation, and evaluation of RaiVeX Library, a comprehensive collection of RISC-V Vector Extension (RVV 1.0) accelerated kernels targeting machine learning and high-performance computing workloads. The research addressed a significant gap in the existing literature concerning the availability of production-ready, thoroughly benchmarked vectorized implementations of fundamental ML primitives on the RISC-V architecture. The library encompasses a diverse range of computational kernels organized into six categories: compute-intensive linear operators (matrix multiplication, convolution, transposed convolution, dense layers), pointwise activation and arithmetic kernels (ReLU, Leaky ReLU, bias addition, tensor addition), statistical and normalization kernels (batch normalization, softmax), spatial reduction kernels (max pooling), tensor indexing and data movement operations (gather, gather elements, scatter elements), and post-processing kernels (non-maximum suppression). Each kernel was implemented in C++ utilizing RVV intrinsics with systematic exploration of different LMUL configurations (M1, M2, M4, M8), enabling fine-grained control over vector register utilization and performance optimization. The architectural design emphasizes modularity and reusability through a low-level vector API abstraction layer that encapsulates common RVV operations including vector loads, stores, reductions, multiply-accumulate, and mask operations. Functional correctness was rigorously validated against ONNX golden references using QEMU emulation, while performance benchmarking on the Ara vector co-processor demonstrated substantial speedups ranging from 4$\times$ to over 70$\times$ compared to scalar baseline implementations. The practical applicability of the library was validated through end-to-end inference implementations of LeNet-5 and Tiny-YOLOv2 neural networks, demonstrating that the vectorized kernels integrate seamlessly into real deep learning pipelines. These results collectively establish that the RISC-V Vector Extension, when properly optimized, constitutes a viable and high-performance alternative for accelerating machine learning inference on resource-constrained embedded platforms, contributing both a practical software artifact and empirical evidence to the growing body of research on open-source hardware architectures for AI acceleration.

\subsection{Future Work}

While RaiVeX Library demonstrates significant performance improvements and practical applicability, several avenues for future research and development remain open. The following subsections outline potential extensions, improvements, and deployment considerations that could further enhance the library's capabilities and broaden its impact.

\subsubsection{Extension to Additional Workload Domains}

The current implementation focuses primarily on machine learning kernels; however, the underlying vector primitives and optimization strategies developed in this work are directly applicable to other computationally intensive domains. Future efforts could extend the library to encompass DSP workloads, including Fast Fourier Transform (FFT), Finite Impulse Response (FIR) and Infinite Impulse Response (IIR) filters, and correlation operations that are fundamental to audio processing, telecommunications, and radar applications. Similarly, scientific computing kernels such as sparse matrix operations, linear algebra decompositions (LU, QR, Cholesky), and numerical integration routines would benefit from RVV acceleration. Image and video processing primitives---including color space conversion, histogram computation, morphological operations, and compression algorithms---represent another natural extension. Such diversification would position RaiVeX Library as a general-purpose high-performance computing library for the RISC-V ecosystem rather than being limited to ML workloads alone.

\subsubsection{Deployment on Physical RISC-V Hardware}

A critical next step involves validating and benchmarking the library on physical RISC-V hardware that supports the RVV 1.0 specification. While the Ara vector co-processor provides valuable performance insights through cycle-accurate simulation, deployment on actual silicon would yield more realistic performance metrics accounting for real-world factors such as memory bandwidth limitations, cache behavior, thermal constraints, and power consumption. Emerging RISC-V processors with vector support, including commercial offerings and development boards, present opportunities for such validation. Hardware deployment would also enable energy efficiency measurements, which are particularly relevant for the embedded and edge computing use cases that RISC-V targets. Furthermore, testing on diverse hardware implementations with varying vector lengths (VLEN) would verify the library's scalability and portability across different RISC-V platforms.

\subsubsection{Enhancement of the Python Interface and Abstraction Layer}

The current Python bindings provide functional access to the vectorized kernels through ctypes wrappers around shared libraries; however, significant improvements could enhance usability and developer experience. Future work should focus on developing a higher-level Pythonic API that abstracts memory management, pointer handling, and data type conversions, enabling users to interact with the library using native NumPy arrays without explicit pointer manipulation. Integration with popular deep learning frameworks such as PyTorch or TensorFlow through custom operator registration would allow seamless incorporation of RVV-accelerated kernels into existing ML workflows. Additionally, implementing automatic kernel selection based on input dimensions and available hardware capabilities, along with comprehensive documentation, type hints, and error handling, would significantly lower the barrier to adoption for researchers and practitioners unfamiliar with low-level systems programming.

\subsubsection{Kernel Optimization and Algorithmic Improvements}

While the implemented kernels demonstrate substantial speedups, further optimization opportunities exist. Advanced techniques such as cache-aware tiling strategies, software pipelining, and loop unrolling specifically tuned for different VLEN configurations could yield additional performance gains. For convolution operations, exploring alternative algorithms such as Winograd-based approaches or more sophisticated im2col-GEMM implementations optimized for specific kernel sizes may prove beneficial. Quantization support for INT8 and INT4 data types, which are increasingly prevalent in edge ML deployment, would enable even greater throughput by leveraging the vector unit's ability to process more elements per instruction. Auto-tuning frameworks that systematically explore the optimization space for given input dimensions and hardware configurations represent another promising direction.

\subsubsection{Expanded Neural Network Model Support}

The successful implementation of LeNet-5 and Tiny-YOLOv2 demonstrates the library's capability to support complete inference pipelines. Future work could expand this to encompass a broader range of neural network architectures, including transformer-based models that have become prevalent in natural language processing and computer vision. Implementing attention mechanisms, layer normalization, and Gaussian Error Linear Unit (GELU) activations would enable support for models such as Bidirectional Encoder Representations from Transformers (BERT), Generative Pre-trained Transformer (GPT) variants, and Vision Transformers. Additionally, developing an ONNX runtime backend that automatically maps supported operators to RVV-accelerated kernels would provide a standardized interface for deploying arbitrary trained models, significantly expanding the library's practical utility in production environments.


% =============================================================================
% References
% =============================================================================
\newpage
\begin{thebibliography}{9}

\bibitem{ourworldindata}
C. Giattino, E. Mathieu, V. Samborska, and M. Roser, ``Data Page: Exponential growth of computation in the training of notable AI systems,'' 
Our World in Data, 2023. [Online]. Available: \href{https://ourworldindata.org/grapher/exponential-growth-of-computation-in-the-training-of-notable-ai-systems}{ourworldindata.org} (accessed: Oct. 2025).

\bibitem{semico2023riscv}
Semico Research Corp., ``RISC-V CPU Market to Exceed 10B Cores by 2030,'' 
Market Report, Semico Research, 2023.

\bibitem{riscv-v-spec}
RISC-V International, ``The RISC-V Vector ISA Extension, Version 1.0,'' 2021. 
[Online]. Available: \href{https://github.com/riscv/riscv-v-spec}{GitHub: riscv-v-spec} (accessed: Jul. 2025).

\bibitem{onnx}
ONNX Project, ``Open Neural Network Exchange (ONNX),'' 2025. 
[Online]. Available: \href{https://onnx.ai/}{onnx.ai} (accessed: Oct. 2025).

\bibitem{vicuna}
Michael Platzer and Peter Puschner. Vicuna: A Timing-Predictable RISC-V Vector Coprocessor for Scalable Parallel Computation. 
In \textit{33rd Euromicro Conference on Real-Time Systems (ECRTS 2021)}. Leibniz International Proceedings in Informatics (LIPIcs), Volume 196, pp. 1:1-1:18,
Schloss Dagstuhl – Leibniz-Zentrum für Informatik (2021) \url{https://doi.org/10.4230/LIPIcs.ECRTS.2021.1}

\bibitem{ara}
Matteo Perotti, Matheus Cavalcante, Nils Wistoff, Renzo Andri, Lukas Cavigelli, and Luca Benini. A "New Ara" for Vector Computing: An Open Source Highly
Efficient RISC-V V 1.0 Vector Processor Design. In \textit{2022 IEEE 33rd International Conference on Application-specific Systems, Architectures and 
Processors (ASAP)}, 2022. \url{https://doi.org/10.1109/ASAP54787.2022.00017}

\bibitem{riscv-gnu-toolchain}
RISC-V International Collaboration, ``RISC-V GNU Compiler Toolchain.'' 
[Online]. Available: \href{https://github.com/riscv-collab/riscv-gnu-toolchain}{GitHub: riscv-gnu-toolchain} (accessed: Jun. 2025).

\bibitem{qemu-riscv}
QEMU Project, ``QEMU: Open Source Machine Emulator and Virtualizer (RISC-V Target).'' 
[Online]. Available: \href{https://www.qemu.org/}{qemu.org} (accessed: Jun. 2025).

\bibitem{verilator}
W. Snyder and contributors, ``Verilator: Open-Source SystemVerilog Simulator.'' 
[Online]. Available: \href{https://www.veripool.org/verilator/}{veripool.org/verilator} (accessed: Jul. 2025).

%%%%%%%%%%%%%%%%%%%%%%%%%
% Papers
%%%%%%%%%%%%%%%%%%%%%%%%%

\bibitem{herdt2018extensible}
V. Herdt, D. Große, H. M. Le and R. Drechsler, "Extensible and Configurable RISC-V Based Virtual Prototype," in \emph{2018 Forum on Specification \& Design Languages (FDL)}, Garching, Germany, 2018, pp. 5--16, doi: 10.1109/FDL.2018.8524047.

\bibitem{schlaegl2024riscv}
M. Schlägl, M. Stockinger and D. Große, "A RISC-V “V” VP: Unlocking Vector Processing for Evaluation at the System Level," in \emph{2024 Design, Automation \& Test in Europe Conference \& Exhibition (DATE)}, Valencia, Spain, 2024, pp. 1--6, doi: 10.23919/DATE58400.2024.10546838.

\bibitem{quiroga2022reusable}
J. Quiroga, R. I. Genovese, I. Diaz, H. Yano, A. Ali, N. Sonmez, O. Palomar, V. Jimenez, M. Rodriguez, and M. Dominguez, "Reusable Verification Environment for a RISC-V Vector Accelerator," in \emph{Proceedings of the Design and Verification Conference \& Exhibition Europe (DVCon Europe)}, Munich, Germany, Dec. 2022, pp. 1--8.

\bibitem{imianosky2024reliability}
C. Imianosky, D. A. Santos, D. R. Melo, F. Viel and L. Dilillo, "Special Session: Reliability and Performance Evaluation of a RISC-V Vector Extension Unit for Vector Multiplication," in \emph{2024 IEEE International Symposium on Defect and Fault Tolerance in VLSI and Nanotechnology Systems (DFT)}, Didcot, United Kingdom, 2024, pp. 1--6, doi: 10.1109/DFT63277.2024.10753524.

\bibitem{alassir2021arrow}
I. Al-Assir, M. M. Yildirim, and U. Y. Ogras, "Arrow: A RISC-V Vector Accelerator for Machine Learning Inference," in \emph{2021 IEEE 32nd International Conference on Application-specific Systems, Architectures and Processors (ASAP)}, 2021, pp. 123--130.

\bibitem{wang2024speed}
Y. Wang, C. Zhang, Z. Wang, and H. Li, "A Scalable RISC-V Vector Processor Enabling Efficient Multi-Precision DNN Inference," in \emph{IEEE Transactions on Very Large Scale Integration (VLSI) Systems}, 2024.

\bibitem{carpentieri2025performance}
L. Carpentieri, M. VazirPanah and B. Cosenza, "A Performance Analysis of Autovectorization on RVV RISC-V Boards," in \emph{2025 33rd Euromicro International Conference on Parallel, Distributed, and Network-Based Processing (PDP)}, Turin, Italy, 2025, pp. 129--136, doi: 10.1109/PDP66500.2025.00026.

\bibitem{volokitin2023case}
V. Volokitin, E. Kozinov, V. Kustikova, A. Liniov, and I. Meyerov, "Case Study for Running Memory-Bound Kernels on RISC-V CPUs," \emph{arXiv preprint} arXiv:2305.09266, 2023, doi: 10.48550/arXiv.2305.09266.

\bibitem{brown2025riscv}
N. Brown, "Is RISC-V ready for High Performance Computing? An evaluation of the Sophon SG2044," \emph{arXiv preprint} arXiv:2508.13840, submitted Aug. 2025, doi: 10.48550/arXiv.2508.13840.

%%%%%%%%%%%%%%%%%%%%%%%%%

\bibitem{lecun1998gradient}
Y. Lecun, L. Bottou, Y. Bengio, and P. Haffner, ``Gradient-based learning applied to document recognition,'' \emph{Proceedings of the IEEE}, vol. 86, no. 11, pp. 2278--2324, 1998, doi: 10.1109/5.726791.

\bibitem{redmon2016yolo}
J. Redmon and A. Farhadi, ``YOLO9000: Better, Faster, Stronger,'' \emph{arXiv preprint} arXiv:1612.08242, 2016, doi: 10.48550/arXiv.1612.08242.

\bibitem{spike-simulator}
RISC-V International, ``Spike: RISC-V ISA Simulator,'' 2024. 
[Online]. Available: \href{https://github.com/riscv-software-src/riscv-isa-sim}{GitHub: riscv-isa-sim} (accessed: Jul. 2025).

\bibitem{cv32e40x}
OpenHW Group, ``CV32E40X RISC-V Core.'' 
[Online]. Available: \href{https://github.com/openhwgroup/cv32e40x}{GitHub: cv32e40x} (accessed: Jul. 2025).

\bibitem{cva6}
F. Zaruba and L. Benini, ``The Cost of Application-Class Processing: Energy and Performance Analysis of a Linux-Ready 1.7-GHz 64-Bit RISC-V Core in 22-nm FDSOI Technology,'' \textit{IEEE Transactions on Very Large Scale Integration (VLSI) Systems}, vol. 27, no. 11, pp. 2629--2640, Nov. 2019.

\end{thebibliography}

% =============================================================================
% Code Listings (Appendix)
% =============================================================================
\newpage
% !TEX root = ../Thesis.tex
\section{Code Listings}

\textit{[Placeholder]}


\end{document}
